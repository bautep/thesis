\section{Solutions to RPA Equations\label{sec:RPAsol}}
The various types of RPA equations, that were introduced in the previous
sections, can be solved with
the same two methods, which work for particle-hole
as well as particle-particle (pp) or hole-hole (hh) RPA. 
The pp and hhRPA are described 
in section~\ref{sect:ppRPA}. Below the two methods, known as
discrete and continuous RPA, are sketched. Due to the similarity of 
the phRPA equation with the ppRPA equation, 
the following equations hold also for 
the two-particle propagator $G^{II}$ instead of 
the polarization propagator $L$.

The RPA equation (\ref{eq:RPA}) can be written in the following notation
%
	\begin{equation}
		L_{ij}(\omega)
	=
		L^0_{ij}(\omega)
	+	
		L^0_{ik}(\omega)
		V_{kl}
		L_{lj}(\omega)
	\label{eq:phRPA}
	\;.
	\end{equation}
%
The indices $i$ and $j$ are introduced to stress the fact that the RPA 
equation is {\em de facto} a matrix equation. 
These indices  represent the configurations of the form $\{\alpha\beta\}$.
This mapping is very effective because of angular momentum coupling, \cf\
appendix~\ref{app:detail}.
The Lehmann representation for $L$ (\ref{eq:DefLLeh}) is written in this 
notation
%
	\begin{equation}
		L_{ij}(\omega)
	=
		\sum_{n \neq 0}
		\frac{ X^n_i X^n_j }
		{ \omega - E^n - i \eta }
	-
		\sum_{m \neq 0}
		\frac{ Y^m_i Y^m_j }
		{ \omega - E^m + i \eta }
	\label{eq:solL}
	\end{equation}
%

\subsubsection{Discrete RPA\label{sec:RPAdisc}}
The RPA equation (\ref{eq:phRPA}) can be solved if it is transformed into an
eigenvalue equation. 
This transformation can be performed,
using the general expression for the polarization propagator
(\ref{eq:solL}) and the explicit form of the mean field polarization propagator
%
	\begin{equation}
		L^0_{ij}(\omega)
	=
		{\sc diag} 
		\left\{
		\frac{ \delta(i,\{hp\})}
		{ \omega - \varepsilon_i + i \eta}
	-
		\frac{\delta(i,\{ph\})}
		{ \omega - \varepsilon_i - i \eta}
		\right\}
	\label{eq:solL0}
	\;.
	\end{equation}
%
Here the notations $\delta(i,\{hp\})=\theta(F-\alpha)\theta(\beta-F)$ and
$\varepsilon_i=\varepsilon^{HF}_\alpha-\varepsilon^{HF}_\beta$ are introduced. 
The
symbol ${\sc diag}$ denotes that $L^0$ is a diagonal matrix in the indices 
$i$ and $j$.
The eigenvalue equation is now obtained by extracting the residues of
the poles of the exact propagator (\ref{eq:solL}). After substitution of
(\ref{eq:solL}) and (\ref{eq:solL0}) into (\ref{eq:phRPA}), multiplication by 
 $(\omega - E^n)$ and having taken the limit 
 $\omega \rightarrow E^n$, the eigenvalue equation to be
solved is
%
	\begin{equation}
		\left(
			\varepsilon 
		+
			V(E^n)
		\right)X^n
	=
		E^n X^n
	\label{eq:RPAeig}
	\;,
	\end{equation}
%
where the mean field energies only appear on the diagonal of the matrix, and
the amplitudes $X$ are taken to be real. In the case of ERPA, where
the (effective) interaction is energy-dependent, (\ref{eq:RPAeig})  is a 
generalized eigenvalue equation, \cf\ (\ref{eq:DysDis}) and should be
solved consistently with respect to $E^n$ (the matrix $\varepsilon +V$ is 
dependent on the eigenvalues $E^n$).
The normalization condition for that case of an RPA equation with 
energy-dependent potential is derived in ref.\cite{HDA86,AEG93}.

\subsubsection{Continuous RPA}
The discrete RPA method has been derived with a single-pole form of $L^0$ 
(\ref{eq:solL0}).
It has to be used, for the matrices involved ($L$ and $L^0$)
 are singular because
of their pole structure. If $L^0$ is not of single-pole form (like in 
Dressed RPA), the dimension of the RPA matrix of (\ref{eq:RPAeig}) will 
grow proportional to the number of fragments in $L^0$.
When  the poles in $L^0$ and $L$ are shifted
away from the real axis by replacing the positive infinitesimal $\eta$ in 
(\ref{eq:solL}) and (\ref{eq:solL0}) by a finite number, the singularity is 
removed.
This will have the effect that 
in the final result (\cf\ eq.~(\ref{eq:SopO})) the delta functions 
will have a finite width, which is justified by the fact, 
that in measurements the energy resolution is always finite, and hence delta
functions will not be observed. 

Having moved the poles away from the real axis, a closed expression for 
 $L$ can be written down. Multiplying (\ref{eq:phRPA}) on the left with
 ${L^0}^{-1}$  and on the right with $L^{-1}$, respectively the 
inverses of $L^0$  and 
 $L$, the RPA equation reads
%
	\begin{equation}
		{L_{ij}(\omega)}^{-1}
	=
			{L^0_{ij}(\omega)}^{-1}
			-
			V_{ij}
	\label{eq:RPAcont}
	\;.
	\end{equation}
%
Because of the finite $\eta$ the matrix to be inverted will be regular but
complex  for real $\omega$'s.
One remark should be made here. A matrix inversion is quite expensive (in the 
sense of algorithmic complexity) and in 
general not all  the matrix elements of $L$ are required. 
In appendix~\ref{app:detail} a method is given to calculate the response
function (\ref{eq:defS}) is given that does not require the calculation of 
the inverse of $(L^0)^{-1} - V$ as an intermediate step.

\subsubsection{Discrete DRPA}
In chapter~\ref{chap:PPO} we will need the transition amplitudes to 
a certain final state of the residual nucleus explicitly, and now we 
face the problem that 
DRPA can not easily be handled with the discrete method described above
because the RPA matrix in (\ref{eq:RPAeig}) will grow proportionally with the
number of poles in the free propagator $L^f$. 
In general $L^f$ will have many poles, \cf~(\ref{eq:drpa:Lf}), 
for the single-particle spectral function is highly fragmented.
The continuous method can be applied (to the two-particle Green's function)
but this method does not yield the 
vectors $X^n_{ij}$ that will be needed to calculate the spectral function
in chapter~\ref{chap:PPO}.

The way around this problem has
already been suggested in ref.\cite{AEG93} in the form of a contour
integration. Inspired by the form of the Lehmann representation of
the propagator (\ref{eq:solL}) the amplitudes $X$ and $Y$
can be determined if the corresponding energies are well separated. In the case
one is interested in amplitudes to the ground state and first few excited states
of $^{14}$C, this separation is 
present. To extract amplitudes that belong to a set of closely lying states
is not possible due to numerical difficulties.
Using (\ref{eq:RPAcont})
as a description of a complex function $L_{ij}(z)$ ($z$ is $\omega$ but 
allowing complex values), the Cauchy 
integral formula can be applied to obtain the amplitudes $X$
%
	\begin{equation}
		X^n_{i} X^{n}_{j}
	=
		\frac{1}{2 \pi i}
		\oint_{C_n} {\rm d} z\;
		L_{ij}(z)
	\;,
	\label{eq:contour}
	\end{equation}
%
where $z$ is a complex variable and $C_n$ is a contour around $E^n$
excluding all other poles of $L$, such that $L(z)$ is regular for $z\in C_n$.
