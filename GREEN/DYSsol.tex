\subsection{Solutions to the Dyson Equation\label{sec:DYSsol}}

The Dyson equation (\ref{eq:Dyson}) is usually solved in an energy 
formulation
%
	\begin{equation}
		g_{\alpha\beta}(\omega)
	=
		g^0_{\alpha\beta}(\omega)
	+
		\sum_{\gamma\delta}
		g^0_{\alpha\gamma}(\omega)
		\Sigma^\ast_{\gamma\delta}(\omega)
		g^{\phantom{0}}_{\delta\beta}(\omega)
	\;.
	\label{eq:DysonE}
	\end{equation}
%
The specific form of the bare propagator $g^0(\omega)$ can be derived from
the Lehmann representation (\ref{eq:g1}), when we consider a system without
residual interaction. In this case we have only the diagonal (one-body) part 
of the
Hamiltonian (\ref{eq:Hamiltonian}) and the ground state of the $A$-body
system as well as the ground and excited states of the ($A-1$)-body system 
will be 
Slater determinants. Hence the bare propagator has the form
%
	\begin{equation}
		g^0_{\alpha\beta}(\omega)
	=
		\delta_{\alpha\beta}
		\left[
		\frac{
		\theta( \alpha - F )
		}{ \omega - \varepsilon_\alpha + i \eta }
	+
		\frac{
		\theta( F - \alpha )
		}{ \omega - \varepsilon_\alpha - i \eta }
		\right]
	\label{eq:g10} 
	\;,
	\end{equation}
%
where the notion of Fermi level is introduced and the step function
$\theta(\alpha-F)$ has the value one (zero) if the index $\alpha$ 
denotes a state above (below) the Fermi level. 

One may now formally obtain the ($A+1$)-part of the propagator from the Dyson 
equation (\ref{eq:DysonE}) by
inserting the form for the bare propagator $g^0$ (\ref{eq:g10}) together with
the full
propagator (\ref{eq:g1}) into the Dyson equation, multiplying it by
$(\omega - (E^{n,A+1}- \EnulA))$ and taking the limit 
$\omega \rightarrow (E^{n,A+1}- \EnulA)$ to extract the residue of the $n$-th
pole for each possible $n$.
The ($A-1$)-part of the propagator can be calculated with a similar procedure
from the equation
%
	\begin{equation}
		\tilde{E}^m  
		X_\alpha^m
	=
		\sum_\beta
		\left(
			\varepsilon_\alpha +
			\Sigma^\ast_{\alpha\beta}(\tilde{E}^m)
		\right)
		X_\beta^m
	\label{eq:DysDis}
	\;,
	\end{equation}
%
where $\tilde{E}^m = (\EnulA - E^{m,A-1})$ and 
$X_\alpha^m=\ME<\Psi^{m,A-1}| \Oa{\alpha} | \Psi^A_0 >$.

The solution of eq.~(\ref{eq:DysDis}) in the approximation that $\Sigma^\ast$
does not depend on the full propagator, can be found by solving this generalized
eigenvalue equation iteratively, combining eigenvalue finding algorithms 
with zero finding procedures.
In the case the self-energy does depend on the full propagator, 
this scheme can be more elaborate.
The solution in both cases will consists of a set eigenvalues and 
eigenvectors. 
The normalization of the eigenvectors can also be obtained from the Dyson
equation (\ref{eq:Dyson}) \cite{Heng} by expanding
the Dyson equation  around $\tilde{E}^m$ to first
order in $(\omega-\tilde{E}^m)$ and using the pole structure of the full 
propagator and  inserting equation (\ref{eq:DysDis}) the normalization of
the eigenvector can be obtained from
%
	\begin{equation}
		\sum_\alpha
		|X_\alpha^m| ^2
	=
		1 
	+
		\sum_{\alpha\beta}
		X_\alpha^m
		X_\beta^m
		\left.
			\frac{ {\rm d} \Sigma^\ast_{\alpha\beta}(\omega)}
			     { {\rm d} \omega}
		\right|_{\omega = \tilde{E}^m}
	\label{eq:DysNorm}
	\;.
	\end{equation}
%
This procedure is similar to the procedure to derive the normalization of the
RPA vectors\cite{AEG93}.

In fig.~\ref{fig:BHF} the lines in the ladder sum correspond to states outside
the model space that will be chosen to do shell model structure calculations
in. This means that in Brueckner-Hartree-Fock, these (high momentum/energy)
states will contribute. It will be shown in chapter~\ref{chap:SPECFAC} that the
effect of these states can be taken into account in a structure calculation 
by considering an 
explicit energy-dependent G-matrix (\ref{eq:BGEG}) instead of a static 
approximation.
