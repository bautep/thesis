\section{Introduction}
Green's functions are the cornerstones of the Feynman diagram techniques in 
many-body perturbation theory. Their field of application contains besides 
nuclear spectroscopy also field theory, solid state physics and statistical 
mechanics. The basic requirement of these techniques is that one can approximate the many-body Hamiltonian by the sum of two parts, one of which has 
relatively simple exact solutions, \eg\ coherent states or (generalized) Slater
determinants. These solutions must be of such a form that they allow the 
definition of normal products of operators and consequently the application of 
Wick's theorem\cite{AGD63,FW71}. The remainder of the Hamiltonian is then 
treated as a perturbation, which is often not small, but parts of it may be 
treated to all orders by the use of integral equations.

The formal development of this many-body perturbation theory and its 
representation with Feynman diagrams may be found in various textbooks
\cite{AGD63,FW71,KE88}. 
In the following sections we shall restrict ourselves to a 
survey of the formalism and the applications that are directly relevant for the 
work described in the following chapters of this thesis.
