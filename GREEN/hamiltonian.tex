\section{Definition of the Nuclear Many-Body Problem}
%\subsection{Hamiltonian}

With the assumption that the nucleus may be considered as an assembly of inert 
nucleons, that interact by two-body interactions only, the nuclear Hamiltonian 
is written in second quantized form as
%
	\begin{equation}
		\hat{H}
	=
		\sum_{\alpha\beta}
		T_{\alpha\beta}
		\Oc{\alpha} \Oa{\beta}
	+
		\sfrac{1}{4}
		\sum_{\alpha\beta\gamma\delta}
		V_{\alpha\beta\gamma\delta}
		\Oc{\alpha}\Oc{\beta}\Oa{\delta}\Oa{\gamma}
	\label{eq:H1}
	\;,
	\end{equation}
%
where the Greek indices indicate a set of quantum numbers that label a certain 
basis set of single-particle states.
Here $T_{\alpha\beta}$ are the matrix elements of the kinetic energy operator 
and $V_{\alpha\beta\gamma\delta}$ those of the interaction in antisymmetrized 
form.
The operators $\Oc{\alpha}$ 
($\Oa{\alpha}$) are the usual creation (annihilation) operators for one 
particle in state $\alpha$.

Empirically it is known that the nucleus is a bound system with features of 
shell structure. Therefore, a logical first step would be to construct a 
single-particle basis with the well-known Hartree-Fock (HF) procedure, which 
is also applied in atomic physics\cite{Ha23,Fo30}. However,  the strong 
short-range repulsion prohibits its straightforward application, because it 
would make the HF mean-field potential repulsive rather than an attractive 
potential well with bound states. In order to obtain a better result one must 
treat the short-range repulsion in a two-body scattering equation to all 
orders, thereby constructing effective interaction matrix elements in analogy 
with the T-matrix for the scattering of free particles.
The procedure to implement this feature into the HF procedure is known as 
Brueckner-Hartree-Fock (BHF)\cite{TFM74} and is diagrammatically represented in 
fig.~\ref{fig:BHF}.
%%%%%%%%%%%%%%%%%
\begin{figure}
\[
\cntrbox{2.2cm}{2cm}{
\epsfig{figure=figures/selfenergy/sig_UHF.ps, width=1.7cm}
}
=
\cntrbox{2.2cm}{2cm}{
\epsfig{figure=figures/selfenergy/sig_UHF1.ps, width=1.7cm}
}
+
\cntrbox{2.2cm}{2cm}{
\epsfig{figure=figures/selfenergy/sig_UHF2.ps, width=1.7cm}
}
+
\cntrbox{2.2cm}{2cm}{
\epsfig{figure=figures/selfenergy/sig_UHF3.ps, width=1.7cm}
}
+
\cdots
\]
\caption[]{ The Hartree-Fock potential must be constructed with a G-matrix 
interaction that solves the two-body scattering in the nuclear medium to all 
orders 
(Brueckner-Hartree-Fock(BHF))\cite{MS93a}.
The wiggly line represents the G-matrix, which is calculated as a scattering
matrix within the medium of the $NN$-interaction (dashed line).
\label{fig:BHF}}
\end{figure}
%%%%%%%%%%%%%%%%%

In the present thesis we do not perform BHF calculations to obtain a 
single-particle basis. Instead, we shall work in a harmonic oscillator basis. 
Besides reasons of convenience, \eg\ the separation of relative and 
center-of-mass motion performed in chapter~\ref{chap:ppknock}, their use may be 
motivated by the fact that for light nuclei like $^{16}$O and $^{48}$Ca 
these wave functions are good approximations of the well-bound, occupied 
states that one obtains in BHF calculations. Then, as a restriction of the 
problem, we shall deal with a space built with only a limited number of 
oscillator shells. The aim is to choose this (shell) model space so large
that is can accommodate the most important long-range correlations, \eg\ those 
that give rise to collective motion (low-energy `phonons', 
Giant Resonances) as well as the coupling of the motion of the individual 
nucleons to these collective phenomena.
The model space will be too small, however, to accommodate short-range 
correlations which are caused by the repulsive core of the $NN$-interaction. 
The latter involves components with large (relative) momenta, which require 
harmonic oscillator states with large quantum numbers, \eg\ up to 
$200 \hbar\omega$ for a Reid soft-core potential\cite{MS93a}.

The practical point of view adopted here, which is also adopted in some 
detail in chapter~\ref{chap:ppknock}, is that the occupied oscillator states 
are a good representation of the BHF states, so that the BHF ladder diagrams 
are already included in the single-particle energies. The latter are then 
calculated by diagonalizing the sum of the kinetic energies and the 
Hartree-Fock 
terms computed with the G-matrix from the Bethe-Goldstone equation
%
	\begin{equation}
		\varepsilon_{\alpha}
	=
		T_{\alpha\alpha}
	+
		\sum_{\gamma<F}
		\left(
			{\rm G}_{\alpha\gamma\alpha\gamma}
		-
			{\rm G}_{\alpha\gamma\gamma\alpha}
		\right)
	\label{eq:SPE}
	\;.
	\end{equation}
%
Because the matrix $\{T+G\}_{\alpha\beta}$
is found to be almost diagonal for the occupied states, the diagonal form
(\ref{eq:SPE}) is adopted in practice. A further rather important assumption is 
that a suitable effective interaction with the model space is already 
obtained by summation of the particle-particle ladders with intermediate 
scattering states outside the model space. The summation of this ladder 
series by means of solving the Bethe-Goldstone equation\cite{MS93a} 
(\cf\ fig.~\ref{fig:BGEG})
%%%%%%%%%%%%%%%%%
\begin{figure}
\[
\underbrace{
\cntrbox{1.5cm}{2cm}{
\epsfig{figure=figures/BGEG/BGEGl.ps, width=1.0cm}
}
=
\cntrbox{1.5cm}{2cm}{
\epsfig{figure=figures/BGEG/BGEGr1.ps, width=1.0cm}
}
+
\cntrbox{1.5cm}{2cm}{
\epsfig{figure=figures/BGEG/BGEGr2.ps, width=1.0cm}
}
+
\cntrbox{1.5cm}{2cm}{
\epsfig{figure=figures/BGEG/BGEGr3.ps, width=1.0cm}
}
+
\cdots
}_{a}
\hspace{0.8cm}
\underbrace{
\left(
\cntrbox{1.8cm}{2cm}{
\epsfig{figure=figures/BGEG/BGEGb.ps, width=1.5cm}
}
\right)
}_{b}
\]
\caption[]{ 
On the left ($a$) the Bethe-Goldstone equation (BGE)(\ref{eq:BGEG}) is depicted.
The wiggly line is the G-matrix which has contributions to all orders in the
$NN$-interaction (dashed line). Only ladder diagrams  with intermediate 
states outside the model space (denoted by double lines) are summed. 
On the right ($b$) a typical diagram not included in the BGE is shown.
\label{fig:BGEG}}
\end{figure}
%%%%%%%%%%%%%%%%%
%
	\begin{equation}
		G(\omega)
	=
		V
	+
		V
		\frac{ \hat{Q} }
		{\omega - \hat{H}_0 + i\eta }
		G(\omega)
	\label{eq:BGEG}
	\end{equation}
%
leads to the Brueckner G-matrix.
In eq.~(\ref{eq:BGEG}) $\hat{Q}$ denotes the Pauli 
operator that prevents scattering into states
of the model space under consideration. The symbol $\hat{H}_0$ is the 
Hamiltonian without the two-body interaction term. 
The positive infinitesimal $\eta$ is
explicitly written down in (\ref{eq:BGEG}) to stress the similarity with
the Lippmann-Schwinger equation that gives the T-matrix of two particles in
vacuum.
The energy-dependence is weak for low-energy spectroscopy and therefore usually
neglected (an average `starting energy' $W$ is adopted). 

The G-matrix is only an approximation of the effective interaction that may be 
formally derived by applying many-body perturbation 
theory\cite{Br67a,JB71,Kuo74,EO77}. 
For instance, the contribution of fig.~\ref{fig:BGEG}b is not included. The 
inclusion of at least the ladder series, given in fig.~\ref{fig:BGEG}a, 
is the 
minimum requirement to obtain well-behaving effective interaction matrix 
elements in the model space.
%
The importance of diagrams like the one shown in fig.~\ref{fig:BGEG}b for 
low-energy spectroscopy may perhaps be reduced by adopting as large a model 
space as feasible.

Summarizing, the many-body problem that we are dealing with is that of the 
Hamiltonian 
%
	\begin{eqnarray}
		\hat{H}
	&=&
		\sum_{\alpha}
		\varepsilon^{\phantom{\dagger}}_{\alpha}
		\Oc{\alpha} \Oa{\alpha}
	+
		\sfrac{1}{4}
		{\sum_{\alpha\beta\gamma\delta}}^\prime
		V^{\phantom{\dagger}}_{\alpha\beta\gamma\delta}
		\Oc{\alpha}\Oc{\beta}\Oa{\delta}\Oa{\gamma}
	\nonumber \\
	&=&
		\hat{H}_0
	+
		\hat{H}_1
	\label{eq:Hamiltonian}
	\;,
	\end{eqnarray}
%
in a large but finite harmonic oscillator 
shell model space and with a given G-matrix interaction. The single-particle 
energies, partly computed with (\ref{eq:SPE}) and partly re-adjusted so as
to reproduce the energies of the quasi-particle peaks for states near the 
Fermi level, are 
supposed to incorporate the (Brueckner-)Hartree-Fock self-energies. Since this 
Hamiltonian is an effective one, it must be kept in mind how it was obtained. 
For instance, when studying short-range correlations in 
chapters~\ref{chap:ppknock} and~\ref{chap:PPO},
one must go one step back to the underlying G-matrix calculation.
Another manifestation of the short-range correlations when using a finite 
shell model space is studied in chapter~\ref{chap:SPECFAC}.
The fact that the G-matrix is energy-dependent through the energy 
denominator in the Bethe-Goldstone equation (BGE)
is crucial here. 

Because of the large size of the model space, which extends over several major 
shells, a straightforward application of the standard shell model 
techniques\cite{DeT63,BG77}, \ie\ construction of a full set of states allowed
by coupling of angular momenta and parity, is out of the question. This is the 
reason why Green's function techniques are adopted. Moreover, one may hope to 
obtain guidance from the transparency of the diagrammatic representation in 
identifying the most relevant parts of nuclear dynamics.
