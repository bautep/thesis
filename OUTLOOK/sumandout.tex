In the various chapters of this thesis, different Green's functions are 
calculated that describe different processes in the nuclear system. 
Investigations are performed in two fields. First, there is the field of 
nuclear physics, where the aim is to understand what the processes are that 
play a major role and what can be learned from experiments on nuclei. 
In the second place, there is the analysis of 
many-body techniques aiming at  a better 
understanding of which equations should be used 
for a certain class of systems and how they can be solved.

The calculation of the Gamow-Teller (GT) response in chapter~\ref{chap:DERPA}
has been performed to check the quality of the Dressed RPA (DRPA) method.
It has been found that the DRPA results are not modified 
drastically by inclusion of medium effects in the interaction in the form of 
a Dressed Extended RPA (DERPA). Therefore, the conclusions of the DRPA 
calculation, \viz\ that a large fraction of 
the GT-strength is mainly present in tails of
the strength distribution that  stretch out to an experimentally unprobed 
energy region, still hold. 
Recently performed shell model calculations\cite{CZPM94,CPS95,LDR95} are by 
construction (including only one major shell) unable to describe the strength 
tails. 
Both the DRPA approach and the shell model calculations 
predict the same strength in the experimentally probed energy region, 
but their description of the strength distribution outside that
region is quite different. Therefore,
more light should be cast on this (energy) region out of reach of the present
experiments, before more effort should be spent in further theoretical work,
such as a refinement
of the approximations in the Bethe-Salpeter equation for the polarization 
propagator that describes this response. 
It would be interesting to extend the measurements of the
GT strength distribution to higher excitation energies and to  perform
shell model calculations with many major shells (truncating the
number of configurations). 

Another interesting application of the research on the polarization
propagator is to study whether 
the fragmentation of single-particle strength has
a similar effect on $T=0$ excitations 
(\ie\ excitations within the same nucleus, such as
electro-magnetic response). The same D(E)RPA methods are directly applicable
to this case.

The subject of the spectroscopic factors of $^{16}$O has been addressed in
chapter~\ref{chap:SPECFAC}. There it has been concluded that the calculated
spectroscopic factors for the low-energy quasi-particle states
may become
in reasonable agreement with experiment when the center-of-mass motion 
is taken into account. 
The obtained fragmentation of strength is still too low.
An  unsatisfactory aspect of the present approach is that 
the calculation is a composition of results from different approaches. 
There is not one consistent
calculation. A possible improvement is to  calculate  
the matrix elements $\ME< \Psi^{A-1}_n \phi_k|\Oc{ }\Oa{ } |\Psi^A_0>$
within the $A$-body system (with excited states where one particle
is free). These matrix elements could be calculated using the 
polarization propagator. 
This may be a way to incorporate the center-of-mass correlations, because then
one could keep track of all the particles and the center-of-mass can be dealt 
with explicitly.

In chapter~\ref{chap:CROSS} the longitudinal structure function in the 
\eepp\ cross-sections in plane-wave approximation has been calculated directly 
from the two-hole spectral function
derived in chapter~\ref{chap:ppknock} and~\ref{chap:PPO}. 
The results reveal that the longitudinal part 
at high relative momenta of the two protons is a factor two larger for the Reid
soft-core potential than calculated with the Bonn potential. 
At very high momenta ($2$--$4$~fm$^{-1}$) also 
the shape of the spectral functions calculated with the different SRC models 
differs, but these momenta can only be probed with electron beams of higher 
energy, \eg\ those that will be available at CEBAF.
One should realize that the spectral function is only an ingredient in the
calculation of the 
full cross-section. Several other mechanisms should be taken into account
that may contribute.
Because of interference effects this should be done in one calculation.
Nevertheless one may try to  estimate these effects. 
The main features
that must be taken into account are: final state interactions (FSI) and
the reaction mechanism (short-range correlations (SRC) 
\vs\ Isobar Currents (IC) and Meson Exchange 
Currents (MEC)). A study of the FSI effects by the Pavia group shows that
the effect of FSI on the total cross-section is mainly a reduction with a 
factor two to three,
depending on the applied optical potential\cite{GP91}. 
So, if one assumes the FSI effect to be a reduction with the average factor 
$0.4$, this gives rise to uncertainties in the analysis of about $20$\%. 
Accurate measurements to make a longitudinal-transverse separation 
of the cross-section will be needed as well, because it is known from 
gamma-induced knock-out reactions that the transverse cross-section has 
large contributions from MEC.
A theoretical study\cite{GP91} of the reaction mechanisms reveals that 
SRC and IC/MEC can be 
probed with measurements at different kinematical settings. The relative 
importance of effects other than SRC effects is not well-known. 
The Pavia group\cite{GP91} has
reported that, under certain kinematical conditions, the IC contribution is 
negligible, while calculations of the Ghent group\cite{Ry94} 
suggest large contributions of IC.
Therefore it will be difficult to discriminate between 
different $NN$-potentials by the \eepp\ cross-sections.
An improved energy resolution will add an extra possibility to compare the 
different SRC models,  because the differences in the calculated
structure functions for the
transition to the $2_1^+$ state are larger than for the $0^+$ ground state of
$^{14}$C.

The main reason why full calculations of the \eepp\ cross-section including
reaction mechanism, realistic nuclear structure and final state effects are
not yet available is that, in comparison 
with calculations for the \eep\ case, the
\eepp\ calculations are far more difficult. The different groups working in this
field do not have expertise on all the aspects of the calculation of the 
total cross-section.
Therefore, the calculation calls for cooperation of several groups
and the production of compatible computer codes.
The participants in this collaboration must be aware that, in the present
effort to 
calculate the \eepp\ cross-section, little attention is paid to the question
how well-suited the model is that describes the knock-out (only one 
proton is hit by the virtual photon, the other leaves the nucleus due to 
correlation effects). Extension of the description presented here to higher 
momenta and 
energies should be done with caution. After all, the starting point of the 
description is that one considers the nucleus as a many-body system 
consisting 
of inert nucleons. At a certain point alternatives, 
such as  momentum transfer to
a six-quark system\cite{Mu86}, may become relevant.
