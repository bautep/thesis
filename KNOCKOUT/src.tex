\section{Inclusion of both Short- and Long-Range Correlations}
\subsection{Necessity to Split up the Calculation in Short- and Long-Range 
Aspects\label{sec:splitting}}

The strong repulsive
core of the nucleon-nucleon interaction is responsible for high-momentum 
components in the nuclear wave function. To describe a system with SRC 
adequately, a very large basis is required for the wave functions, otherwise
high momenta cannot be described.
Such a large basis cannot be handled computationally, so a different approach
should be taken. 

This new approach is an extension of the methods previously 
used\cite{Br90,Rij93}, \viz\ now short-range correlation(SRC) effects are 
included
in a calculation in which the long-range correlations are treated with a 
G-matrix as effective interaction.
An advantage of splitting the correlations into long-range and 
short-range is that one may compare 
different models for short-range correlations.

The expansion of the two-nucleon removal amplitude in an angular-momentum 
coupled shell model basis 
%
	\begin{eqnarray}
	\lefteqn{%
		\ME< \Psi^{A-2}_{J M}| \Oa{\rvec{p}_1} \Oa{\rvec{p}_2} %
		| \Psi^A_0 >
	\rightarrow} \nonumber \\
	&&
		\frac{(-)^{J-M}}
		     {\sqrt{2J+1}}
		\sum_{ab}
		\RME< \Psi^{A-2}_{J}| 
		(\Oa{\tilde{\alpha}} \Oa{\tilde{\beta}})_J %
		| \Psi^A_0 >
		\Phi_{ab}^{J, -M}(\rvec{p}_1, \rvec{p}_2)
	\label{eq:MEaa}
	\end{eqnarray}
%
that was used to derive (\ref{eq:SpecFuncW})
is still exact, but the summation is in principle infinite. 
In the 
approximation of a mean field ground state, however, the summation is finite
because
the basis states are either completely filled or completely empty, so
all the amplitudes concerning empty orbits will vanish. 
In order to
get a good description of the high-momentum components in the wave function,
one has to take into account states until at least $100\hbar\omega$ for
$^{16}$O\cite{MS93}.

Due to correlations
the shells that were filled in mean field will be partially depleted, while
empty shells will get some occupation. This phenomenon has been observed
experimentally very clearly, \eg\ in one-proton knock-out \eep\
experiments\cite{HBJ88,Ste91,Qu88a,Kr90}. In previous
calculations\cite{BRM91,RAD92} it was shown
that the long-range correlations (represented by the coupling of the particle
to particle-hole phonons or giant resonances) 
cause fragmentation of single-particle strength
and hence depletion of `filled' orbits. In these calculations it appeared to 
be necessary to incorporate also 
SRC in the form of an extra (overall) depletion of about 10\%.
In that case the major effect of SRC seemed to be the uniform depletion of 
the shells under study, shifting the one-particle strength to very
high energies corresponding to orbits far above the Fermi level. In turn
these orbits
contain the necessary high-momentum 
components in the total wave function as the effect of SRC.
 It is
therefore meaningful to
formally split the model space into a part $\calM$ (\eg\ four
or five oscillator shells in $^{16}$O) that will be used to 
calculate the structure due to long-range correlations and the complementary
space $\calbarM$ of which the orbits contribute to the spectral function
as a consequence of the SRC.

As a next step an effective interaction is determined for the calculation
of the amplitudes within the model space $\calM$. This interaction will of 
course depend on
the choice of $\calM$. In principle such an interaction is well-defined%
\cite{Br67a,JB71,Kuo74,EO77} but very complicated, 
\eg\ containing 
three- and more-body parts. One may argue that
SRC cause the major difference between the effective and original 
$NN$-interaction, and SRC are supposed to have the property that they can
be well-described at the two-body level. Namely, SRC are effects
caused by close encounters of two nucleons, and considering the range
of $0.5$~fm of the short-range forces as a typical measure for the size of
the nucleons, the density of nucleons is low. 
This is the basis for the
individual pair approach of Brueckner \etal\cite{BGW58,BLR61}.

\subsection{Brueckner G-matrix Approach}
In order to be able to incorporate all relevant diagrams the diagrams are
summed in two stages. The diagrams that involve short-range effects are summed 
in the Bethe-Goldstone equation (BGE) (\ref{eq:BGEG}) for the G-matrix. Another
part of the 
series for the two-particle Green's function  is then summed in the 
Bethe-Salpeter equation (BSE) (\ref{eq:BSEpp}) using the G-matrix as an 
effective 
interaction within a limited model space. The BGE is a two-body scattering 
equation for an individual pair while the presence of the other particles is 
only accounted for by the average binding energy of the pair in the mean field
potential and by the Pauli principle, which forbids as intermediate states in 
the scattering process the 
states of the chosen model space.
The G-matrix is the analogue of the T-matrix in the  Lippmann-Schwinger
equation. The procedure to calculate the G-matrix is sketched below.

The starting point is to truncate the basis for the single-particle states,
thus obtaining a smaller
set ($\calM$) of basis states that are taken into account in $\calM$. 
Once a truncated set is defined,
the model
space for states coupled to total angular momentum $J$ and projection $M$ is
described by a set $\ket{\Phi^{JM}_{ab}}$ with 
 $(ab) \in \calM$ (meaning ($a\in\calM$ $\wedge$ $b\in\calM$)). 
In the BGE 
a projection operator $P$ is introduced that projects a state on the chosen
truncated model space. Its complementary projection $\hat{Q}$ is called
the Pauli operator, because its function in the BGE is to prevent scattering
to occupied states. The projection operators can be written as
%
	\begin{eqnarray}
		\hat{P}
	&=&
		\sum_{ab \in \calM}
		\ket{ \Phi_{ab} }
		\bra{ \Phi_{ab} }
	\;,
	\nonumber \\
		\hat{Q}
	&=&
		\hat{1}-\hat{P}
	\;.
	\end{eqnarray}
%
The BGE for the correlated pair wave function $\ket{\Psi^{JM}_{ab}}$ is 
now written as
%
	\begin{equation}
		\ket{\Psi_{ab}}
	=
		\ket{\Phi_{ab}}
	+
		\frac{\hat{Q}}
		{W - \hat{H}_0 + i\eta}
		\hat{V}
		\ket{\Psi_{ab}}
	\label{eq:BGEP}
	\;,
	\end{equation}
%
where $W$ is the starting energy of the propagating pair, which is negative for
a bound system. In (\ref{eq:BGEP}) the symbol $\hat{H}_0$ is the Hamiltonian 
without the residual interaction and $\eta$
is a positive 
infinitesimal. The symbol $\eta$ which is present in (\ref{eq:BGEP}) because
of correspondence with the Lippmann-Schwinger equation where it is relevant,
can be set to zero in this case because the denominator will always be 
negative.
{}From (\ref{eq:BGEP}) it is clear that the wave functions are modified
due to the scattering that involves higher orbits (not belonging to $\calM$).
This is precisely the mechanism with which the short-range correlation will 
be taken into account. Defining the `defect function' $\chi$  as the 
difference between the correlated and uncorrelated wave function, so
%
	\begin{equation}
		\ket{\chi}
	=
		\frac{\hat{Q}}{W - \hat{H}_0 +i\eta}\hat{V} \ket{\Psi}
	\label{eq:def}
	\;,
	\end{equation}
%
the effect of SRC can be included by the replacement of the uncorrelated
basis functions by correlated ones
%
	\begin{equation}
		\ket{\Phi}
	\rightarrow
		\ket{\Psi}
	= 
		\ket{\Phi} + \ket{\chi}
	\label{eq:def2}
	\;.
	\end{equation}
%

The BGE for the G-matrix can be obtained from the definition of the G-matrix
which is written, analogous to the definition of the T-matrix, as
%
	\begin{equation}
		\ME<\Phi^{JM}_{cd}| G |\Phi^{JM}_{ab}>
	=
		\ME<\Phi^{JM}_{cd}| V |\Psi^{JM}_{ab}>
	\;.
	\end{equation}
%
In the form of an operator equation the BGE then reads
%
	\begin{equation}
		G
	=
		V + V \frac{Q}{W - H_0 +i\eta}G
	\label{eq:BGG}
	\;.
	\end{equation}
%
At this point the modification of eq.~(\ref{eq:SpecFuncW}) by the G-matrix 
approach, \ie\ the truncation of the model space and the introduction of
correlated wave functions can be expressed as
%
	\begin{eqnarray}
	\lefteqn{%
		S^{hh}_J( \rvec{p}_1, \rvec{p}_2, 
		          \rvec{p}_{1'}, \rvec{p}_{2'}, \omega )
	=
		%\sum_{M}
	} && 
	\label{eq:SpecFuncC}\\
	&&
		\sum_{n,M}
		\sum_{\begin{array}{c}
			ab \in \calM;\\[-8pt]
			cd \in \calM
		      \end{array}}
		\Psi^{\ast JM}_{cd}( \rvec{p}_{1'}, \rvec{p}_{2'} )
		X^{n\ast}_{cd J} X^{n}_{ab J} 
		\Psi^{JM}_{ab}( \rvec{p}_1, \rvec{p}_2 )
		\;
		\delta(\omega - E^{n,A-2}_{J} )
	\nonumber
	\;,
	\end{eqnarray}
%
where $\Psi$ is the correlated two-body wave function, \cf~(\ref{eq:BGEP}).

In general SRC will have effect only on the relative motion of two colliding
nucleons. 
The next step in the calculation of the spectral function therefore is to 
perform the 
transformation to relative and center-of-mass (\CM) coordinates. 

\subsection{Transformation to Relative and Center-of-Mass 
Coordinates\label{sec:Mos}}
In the case one is working with a harmonic oscillator basis the 
transformation to relative and center-of-mass coordinates can be done 
elegantly, as was shown by Talmi\cite{Ta52} and Moshinsky\cite{MB60}.
Although the transformation can always be performed, in the case of a
harmonic oscillator basis for the absolute motion, the relative and \CM\ 
motion can also be described with harmonic oscillators. Furthermore the 
transformation brackets can be written down in closed form.

If a product of two harmonic oscillator wave functions with arguments
$\rvec{r}_1$ and $\rvec{r}_2$ and harmonic oscillator constant
%
	\begin{equation}
		b
	=
		\sqrt{\frac{\hbar}
		           {m\Omega}
		}
	\label{eq:bho}
	\end{equation}
%
is expressed in
relative ($\rvec{r}_{rel}=\rvec{r}_1-\rvec{r}_2$) and \CM\ 
($\rvec{R}_{\CM}=\half(\rvec{r}_1+\rvec{r}_2)$) coordinates the corresponding 
constants are
 $b_{rel}=\sqrt{2}b$ (because the relevant mass in (\ref{eq:bho}) is 
the reduced mass of the system $\mu=\half m$) 
and $b_{\CM}=\half\sqrt{2}b$ (here the relevant mass is
the total mass of the system $M=2m$).
A Fourier transformation of this wave function yields harmonic oscillator
functions with arguments
$\rvec{k}_{rel}=\half(\rvec{k}_1-\rvec{k}_2)$ and 
$\rvec{K}_{\CM}=\rvec{k}_1+\rvec{k}_2$.
The factors $\sqrt{2}$ can be put in more symmetric form if one defines the
relative and \CM\ coordinates and momenta as:
%
	\begin{eqnarray} 
		\rvec{R}_{\CM}  = \frac{\rvec{r}_1+\rvec{r}_2}{\sqrt{2}} \;,& &
		\rvec{r}_{rel} = \frac{\rvec{r}_1-\rvec{r}_2}{\sqrt{2}} \;,\\
		\rvec{P}_{\CM}  = \frac{\rvec{p}_1+\rvec{p}_2}{\sqrt{2}} \;,& &
		\rvec{p}_{rel} = \frac{\rvec{p}_1-\rvec{p}_2}{\sqrt{2}} \;.
	\label{eq:momrelcm}
	\end{eqnarray} 
%
With these definitions all harmonic oscillator functions have the same 
harmonic oscillator constant $b$.

Having defined the relative and \CM\ momenta with (\ref{eq:momrelcm}),
the Talmi-Moshinsky transformation is defined in the following way:
%
	\begin{eqnarray}
	\lefteqn{
		\inp< \rvec{p}_1 \rvec{p}_2 | n_1 l_1 n_2 l_2; \lambda>
	=}
	&&
	\label{eq:MTapply} \\
	&&
		\sum_{nlNL}
		\inp< n l N L; \lambda| n_1 l_1 n_2 l_2; \lambda>
		\inp< \rvec{P}_{\CM} \rvec{p}_{rel}| n l N L; \lambda >
	\nonumber
	\;,
	\end{eqnarray}
%
where the wave functions are coupled in the $(\lambda S)_J$ coupled scheme
%
	\begin{equation} 
		\rvec{l}_1 + \rvec{l}_2 = \gvec{\lambda} ;\;\; 
		\rvec{s}_1 + \rvec{s}_2 = \rvec{S} ;\;\; 
		\gvec{\lambda} + \rvec{S} = \rvec{J} 
	\label{eq:lambdaSJ}
	\;.
	\end{equation} 
%
The transformation brackets have been tabulated by Brody and 
Moshinsky\cite{MB60}. 
In (\ref{eq:MTapply}) the summation is finite due to the energy 
conservation condition of the harmonic oscillators involved
%
	\begin{equation}
		2(n_1+n_2) + l_1+l_2
	=
		2(n+N)+l+L
	\label{eq:MTEC}
	\;.
	\end{equation}
%
A closed form of the brackets in (\ref{eq:MTapply}) is given
in appendix~\ref{app:MTCF}, this is the form that can easily by used in the 
calculations.
	
%\subsubsection{tensor correlations}
%The Moshinsky transformation is valid for uncorrelated wave
%functions. The correlated wave functions do not form a complete basis anymore,
%\eg\ they may all be zero at short relative distances.
%Furthermore, due to tensor interaction, different $l$ values can mix
%and the correlated wave function do not have a definite 
%$l$-value, yielding a problem when one wants to apply eq.~(\ref{eq:MTapply}).
%The problem of the incomplete basis is a formal problem. Basis states with
%suppression of short distances will be able to describe the total correlated
%wave function, which itself will have suppression at short distances. Because
%of the problem of ill-defined $l$ values, a choice has to be made. A possible
%choice is for instance, that in the Moshinsky transformation, 
%the $l$ of the correlated function is taken 
%to be the $l$ of the uncorrelated function, ignoring the mixing effect of the
%tensor correlations in this transformation.

\subsection{Correlation Functions\label{sec:cf}}
Although the addition of defect functions is a straightforward method to 
introduce
the effect of SRC we will take the opportunity to look into another method
in  which correlation functions are used
to describe SRC and which is inspired by the Correlated Basis 
Function (CBF) method.
 In that method the correlated wave
function is related to the uncorrelated ($A$-body) wave function in the 
following way%
\cite{GP94,Ja55,FP88,BFF94}
%
	\begin{equation}
		\Psi^A(\rvec{r}_1,\ldots,\rvec{r}_A)
	=
		\sum_k
		\prod_{i<j}
		f_k(|\rvec{r}_i-\rvec{r}_j|)
		O_k
		\Phi^A(\rvec{r}_1,\ldots,\rvec{r}_A) 
	\label{eq:corrfunctotal}
	\;,
	\end{equation}
%
where $f_k$ are the correlation functions and $O_k$ represent relevant 
operators.
At the level of two-body wave functions one has
%
	\begin{equation}
		\Psi(\rvec{r}_1,\rvec{r}_2)
	=
		\sum_k
		f_k(|\rvec{r}_1-\rvec{r}_2|)
		O_k\Phi(\rvec{r}_1,\rvec{r}_2) 
	\label{eq:corrfunc}
	\;.
	\end{equation}
%
The correlation functions and operators can be determined in a variational
calculation. This has already been done some time
ago for the simplest case\cite{Cl81}
 with only the unity operator. 
The correlated function then is of Jastrow type\cite{Ja55}. 
In general all the 
different operators
that are present in the nuclear interaction should be included in 
(\ref{eq:corrfunc})\cite{FP88}. 
These operators are listed in table~\ref{tab:operators}.
The Pavia group\cite{GP94} has used
correlation functions of Jastrow type
calculated by Clark\cite{Cl81} with a 
variational calculation for two different interactions.

In recent years methods to handle correlated basis functions (CBF) have been 
developed further\cite{FP88}, and applied to ground state properties 
such as momentum distributions\cite{BFF94} and spectral functions%
\cite{BFF89,BFF92} in nuclear matter. Variational calculations of 
similar nature were also carried out with Fermi hypernetted chain theory 
(FHNC), which 
yielded equivalent results for $^{16}$O\cite{CFF94}. 
%For the calculation of 
%the two nucleon spectral function as a function of excitation energy, these 
%variational methods seem not very practical. 
The CBF theory with state-dependent interactions is hard to apply in finite 
systems for the calculation of one- or two-body spectral functions.
A recent calculation of the 
spectrum of $^{16}$O with the CBF method yielded rather poor
results\cite{Bos94}.
%%%%%%%%
\begin{table}
\centerline{
\begin{tabular}{|c|c|}
	\hline
		1 
	&   
		$\gvec{\tau}_1\cdot\gvec{\tau}_2$  
	\\ \hline
		$\gvec{\sigma}_1\cdot\gvec{\sigma}_2$ 
	&
		$\gvec{\sigma}_1\cdot\gvec{\sigma}_2
		(\gvec{\tau}_1\cdot\gvec{\tau}_2)$
	\\ \hline
		$S_{12}$ 
	& 
		$S_{12}(\gvec{\tau}_1\cdot\gvec{\tau}_2)$ 
	\\ \hline
		$\rvec{l}\cdot\rvec{S}$ 
	& 
		$\rvec{l}\cdot\rvec{S}(\gvec{\tau}_1\cdot\gvec{\tau}_2)$ 
	\\ \hline
\end{tabular}
}
\caption{
The operators that are present in the nuclear interaction.
The tensor operator $S_{12}$ is  defined as
$3(\gvec{\sigma}_1\cdot\hat{\rvec{r}}_{12})
  (\gvec{\sigma}_2\cdot\hat{\rvec{r}}_{12})-
  \gvec{\sigma}_1\cdot\gvec{\sigma}_2$.
\label{tab:operators}
}
\end{table}
%%%%%%%%%
