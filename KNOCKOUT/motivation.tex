\section{Motivation}
The main motivation to study the two-nucleon (removal) spectral function
is the role that it is expected to play in two-nucleon knock-out processes, 
notably two-proton knock-out by an electron \eepp. 
Such experiments have 
been proposed\cite{Prop} and are being carried out with the hope
to explore nucleon-nucleon correlations in finite nuclei. 
Especially coincidence measurements involving protons with large momenta might 
provide information on short-range correlations (SRC). The reasoning runs as 
follows\cite{GP94}.
Suppose that the scattered electron 
transfers a virtual photon to one of two strongly correlated protons. This 
strong correlation should be viewed as induced by
hard collisions due to the strong 
repulsive core of the nucleon-nucleon ($NN$) interaction. At the instant
of interaction with the virtual photon the two correlated
protons have large but opposite momenta, so that their relative momentum is
high but the center-of-mass momentum is small. If one of the protons is 
removed
by the absorption of a virtual photon, under the assumption that the 
energy transfer is mainly to the hit pair (the residual nucleus stays at a 
low excitation energy), its partner will also leave the nucleus 
as well. 
%\margin{only at low energy}
It is clear that, if the coupling of the virtual photon to one nucleon
is the dominant mechanism, the
\eepp\ process may present a useful tool to
 investigate short-range correlation effects.

One should keep in mind that this is not the only possible scenario for a 
\eepp\
process. Various other mechanisms are possible.
The virtual photon could couple directly to the mesons that are exchanged 
between the (strongly correlated) nucleons. This process will be suppressed 
in the case of two proton 
knock-out; the virtual photon will not couple to uncharged mesons.  In the case
of proton-neutron knock-out, however, this process may be quite 
important\cite{Ry94}.
Virtual photons can
also cause the formation of a delta excitation, or can couple to an already
existing delta in the nucleus. 
%This process will be suppressed when  measurements are done 
%in the `dip region', between the quasi-elastic and the delta 
%peak\cite{Kes93}. 
In order to avoid that these processes dominate the cross-section, 
experiments should be performed at low missing energies for the 
($A-2$)-system.
Yet another mechanism that will contribute to the total \eepp\ cross-section
is the knock-out of a proton neutron pair followed by a charge exchange of the
neutron. 
Recent calculations\cite{GP95} indicate that, under suitable conditions,
this contribution to the cross-section is small.

In general there is a limitation on probing 
the nucleus with high momenta. The nucleons themselves are 
composite objects and their internal structure will play a more important 
role at higher energies. The delta resonance is the first manifestation of 
this internal structure and one has to be aware of its presence. Still 
short-range correlations are expected to provide the dominant contribution to 
the presence of high-momentum nucleons at low missing energy.
