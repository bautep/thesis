\section{Connection to Green's Function}
An important ingredient in the description of the two-nucleon knock-out 
reaction is the two-hole spectral function that is defined as
%
	\begin{eqnarray}
	\lefteqn{%
		S^{hh}(\rvec{p}_1, \rvec{p}_2, \rvec{p}_{1'}, \rvec{p}_{2'}, 
		\omega )
	=} 
	\label{eq:SpecFunc}
	\\
	&&
		\sum_n
		\ME<\Psi^A_0| \Oc{\rvec{p}_{1'}} \Oc{\rvec{p}_{2'}}|%
		\Psi^{n,A-2} >
		\ME<\Psi^{n,A-2}| \Oa{\rvec{p}_1} \Oa{\rvec{p}_2}|%
		\Psi^A_0 >
		\delta( \omega - (E^{0,A} - E^{n,A-2}) )
	\nonumber
	\;,
	\end{eqnarray}
%
where $\Psi^A_0$ denotes the ($0^+$) ground state of the target system 
 and $\Psi^{n,A-2}$ denotes the $n$-th excited state
of the residual nucleus. The restriction of a $0^+$
target system is not necessary, but is imposed here because the results will
be simpler and we intend to apply the formalism to $^{16}$O, which has a
 $0^+$ ground state. In (\ref{eq:SpecFunc}), 
 $\Oc{\rvec{p}}$ ($\Oa{\rvec{p}}$) are
the creation (annihilation) operators of a nucleon with momentum $\rvec{p}$
(spin and isospin are implicitly assumed).

Since the nuclear wave functions have well-defined angular momentum and 
parity quantum numbers, it is useful to expand the operators also into some 
basis with shell model quantum numbers.
%
	\begin{equation}
		\Oa{\rvec{p}}
	=
		\sum_{\alpha}
		\phi^{\phantom{\dagger}}_\alpha(\rvec{p})
		\Oa{\alpha}
	\label{eq:basisO}
	\;,
	\end{equation}
%
with $\alpha=\{n_\alpha, l_\alpha, j_\alpha, m_\alpha\}$ and
%
	\begin{equation}
		\phi^{\phantom{\dagger}}_\alpha(\rvec{p})
	=
		\inp< \rvec{p} m_s
		|n_\alpha, l_\alpha, j_\alpha, m_\alpha>
	\;.
	\end{equation}
%
 For
the description of bound systems one can use single-particle wave functions 
of, for instance, 
Saxon-Woods or harmonic oscillator form.
With a sufficiently good energy resolution the cross-sections for individual
low-lying 
states, with well defined angular momenta,  may be identified. Therefore, it is 
natural to introduce a pair wave function
in angular momentum coupled form\cite{Ed57,BG77,FW71}
%
	\begin{equation}
		\Phi^{JM}_{cd}( \rvec{p}_{1'}, \rvec{p}_{2'} )
	=
		\sum_{m_\gamma m_\delta}
		\CG( j_\gamma, m_\gamma, j_\delta, m_\delta, J, 
		\; M )
		\phi_\gamma( \rvec{p}_{1'})
		\phi_\delta( \rvec{p}_{2'})
	\label{eq:PhiCoup}
	\;,
	\end{equation}
%
where a Clebsch-Gordan coefficient is employed, and the indices $c$ and $d$ 
denote basis states without the projection quantum number $m$:
$a=\{n_a, l_a, j_a\}$. The spectral function (\ref{eq:SpecFunc}) for final 
states with angular momentum $J$ can now 
be written as
%
	\begin{eqnarray}
	\lefteqn{
		S^{hh}_J( \rvec{p}_1, \rvec{p}_2, \rvec{p}_{1'}, \rvec{p}_{2'}, 
		\omega )
	=}
	&&
	\label{eq:SpecFunc1} \\
	&&
		\sum_{abcd,M}
		\Phi^{\ast JM}_{cd}( \rvec{p}_{1'}, \rvec{p}_{2'} )
		S^-_{cdab;JM}(\omega)
		\Phi^{JM}_{ab}( \rvec{p}_1, \rvec{p}_2 )
	\nonumber
	\;.
	\end{eqnarray}
%
The two-nucleon removal spectral function $S^-_{abcd;JM}$ may be calculated by 
constructing shell model wave functions for the initial and final nucleus and 
next calculating the matrix element of the angular momentum coupled 
two-nucleon removal operator $(\Oa{\rvec{p}_1} \Oa{\rvec{p}_2})_{JM}$.
A more direct method, which we prefer, is to use the relation between
the spectral function and the (imaginary part of the) two-particle Green's 
function
%
	\begin{equation}
		S^-_{abcd;JM}(\omega)
	=
		\frac{-1}{\pi} {\rm Im} \; G^{II}_{abcd;JM}(\omega)
	\;,
	\end{equation}
%
as discussed in section~\ref{sect:ppRPA}.
It describes the propagation of a pair of nucleons through the nuclear medium 
and is given in Lehmann representation and using the angular momentum coupled 
form 
%
	\begin{eqnarray}
	\lefteqn{%
		G^{II}_{abcd;J}(\omega)
	=} 
	\label{eq:g2} 
	\\
	&&
		\sum_{n}
		\frac{
		\RME<\Psi^A_0| (\Oa{\tilde{\beta}} \Oa{\tilde{\alpha}})_J|%
		\Psi^{n,A+2}_J >
		\RME<\Psi^{n,A+2}_{J}| (\Oc{\gamma} \Oc{\delta})_J|%
		\Psi^A_0 >
		}{ \omega - (E^{n,A+2}_{J} - \EnulA) + i \eta }
	\nonumber \\
	&&
	-
		\sum_{m}
		\frac{
		\RME<\Psi^A_0| (\Oc{\gamma} \Oc{\delta})_J|%
		\Psi^{m,A-2}_J >
		\RME<\Psi^{m,A-2}_{J}| 
			(\Oa{\tilde{\beta}} \Oa{\tilde{\alpha}})_J|%
		\Psi^A_0 >
		}{ \omega - (\EnulA - E^{m,A-2}_{J}) - i \eta }
	\nonumber \\
	&&
	=
		\sum_{n}
		\frac{
		Y^{n\ast}_{ab J} Y^{n}_{cd J}
		}{ \omega - (E^{n,A+2}_{J} - \EnulA) + i \eta }
	-
		\sum_{m}
		\frac{
		X^{m\ast}_{cd J} X^{m}_{ab J}
		}{ \omega - (\EnulA - E^{m,A-2}_{J} ) - i \eta }
	\nonumber
	\;.
	\end{eqnarray}
%
The symbols $\RME<\ldots|\ldots|\ldots>$ represent the reduced matrix elements%
\cite{Ed57,BG77,FW71}
of the two-nucleon removal and addition tensor operators that
are constructed by the angular momentum coupling of two one-nucleon addition
and removal tensors ($\Oc{\alpha}$ and $\Oa{\tilde{\alpha}}$, where
$\Oa{\tilde{\alpha}}=(-)^{j_\alpha-m_\alpha}\Oa{-\alpha}$ and
 $-\alpha$ denotes $\{n_\alpha, l_\alpha, j_\alpha, -m_\alpha\}$, the 
time reverse of $\alpha$).
The working expression of the spectral function (\ref{eq:SpecFunc}) 
then becomes 
%
	\begin{eqnarray}
	\lefteqn{%
		S^{hh}_J( \rvec{p}_1, \rvec{p}_2, 
		          \rvec{p}_{1'}, \rvec{p}_{2'}, \omega )
	=
		%\sum_{M}
	} && 
	\label{eq:SpecFuncW}
	\\
	&&
	\hspace{-0.5cm}
		\sum_{n}
		\sum_{abcd,M}
		\Phi^{\ast JM}_{cd}( \rvec{p}_{1'}, \rvec{p}_{2'} )
		X^{n\ast}_{cd J} X^{n}_{ab J} 
		\Phi^{JM}_{ab}( \rvec{p}_1, \rvec{p}_2 )
		\;
		\delta(\omega - (\EnulA -E^{n,A-2}_{J}) )
	\nonumber
	\;.
	\end{eqnarray}
%
The reason for preferring this method is that the Green's function 
(\ref{eq:g2}) can be calculated by means of a Feynman-Dyson expansion, 
derived from the Gell-Mann and Low theorem\cite{FW71,AGD63}. 
Summation of infinite subclasses 
of diagrams makes
it possible to derive approximate equations for the two-particle
removal amplitudes $X$. 
These procedures have been given in section~\ref{sec:lrcBSE}.
%
A direct calculation of the amplitudes $X$, \ie\ performing a shell model 
calculation that takes into account both the coherence effects due to 
long-range correlation and high-momentum mixing effects of the short-range
correlations, will require an excessive model space. By the Green's function 
method one is able to handle larger model spaces than in shell model
calculations. 
%

The summation in (\ref{eq:SpecFuncW}) runs over an infinite set of 
single-particle basis states $abcd$, in principle. 
In order to avoid that the calculation of the amplitudes $X$ becomes impossible,
a truncation of the model space is 
necessary, but the presence of short-range correlation requires a large 
space to describe high-momentum components in the wave functions. 
For this reason a distinction is made between treatment of short- and 
long-range correlations. This approach is presented in the next section.
