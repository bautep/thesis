\section{Two-Nucleon Spectral Function; Computational Scheme}
In this section explicit formulas are derived for the spectral function
(\ref{eq:SpecFuncW}).
As mentioned in section~\ref{sec:Mos} the Moshinsky brackets that will 
be used are between $(\lambda S)_J$ two-body coupled wave functions, while
the wave functions used in (\ref{eq:PhiCoup}) have been coupled in the 
 $(j_aj_b)_J$ scheme
%
	\begin{equation} 
		\rvec{l}_a + \rvec{s}_a = \rvec{j}_a ;\;\; 
		\rvec{l}_b + \rvec{s}_b = \rvec{j}_b ;\;\; 
		\rvec{j}_a + \rvec{j}_b = \rvec{J}. 
	\end{equation} 
%
During the route to derive the explicit formula for the spectral function, one
has to perform a recoupling at one point. This recoupling is given by
%
	\begin{eqnarray} 
	\lefteqn{%
		\Phi^{JM}_{ab}(\rvec{p}_1,\rvec{p}_2) 
	= 
		\inp<\rvec{p}_1 \rvec{p}_2| (l_a s_a) j_a (l_b s_b) j_b \; J M>
	} \nonumber \\
	&=&
		\sum_{\lambda S}
		\inp<(l_a l_b) \lambda (s_a s_b)S \; J M |%
		               (l_a s_a) j_a (l_b s_b) j_b \; J M >
	\nonumber \\
	&\times&
		\inp< \rvec{p}_1 \rvec{p}_2 | 
                      n_a n_b (l_a l_b) \lambda (s_a s_b) S \; J M >
	\label{eq:phiab}
	\;.
	\end{eqnarray} 
%
The last line of (\ref{eq:phiab}) can be written explicitly in case
the one-body wave functions are known as:
%
	\begin{eqnarray} 
	\lefteqn{
		\inp< \rvec{p}_1 \rvec{p}_2 | 
                      n_a n_b (l_a l_b) \lambda (s_a s_b) S J M >
	=} 
	\label{eq:coup1}
	\\
	&&
		\sum_{ m_a, m_b, \mu, s^z_a, s^z_b, S_z }
		\CG( l_a, m_a, l_b, m_b, \lambda, \mu )
		\CG( s^{\phantom{z}}_a, s^z_a, s^{\phantom{z}}_b, 
		     s^z_b, S, S_z)
		\CG( \lambda, \mu, S, S_z, J, M )
	\nonumber \\
	&\times&
		R_{n_a l_a}(|\rvec{p}_1|)
		Y_{l_a m_a}(\hat{\rvec{p}}_1)
		\xi_{s s^z_a}
		\;
		R_{n_b l_b}(|\rvec{p}_2|)
		Y_{l_b m_b}(\hat{\rvec{p}}_2) 
		\xi_{s s^z_b}
	\nonumber
	\;,
	\end{eqnarray} 
%
where $R_{nl}$ are the radial wave functions and $Y_{lm}$ are the 
spherical harmonics. The symbols $\xi$ denote the spin wave functions.
In the present application wave functions are always 
calculated in complex conjugated pairs, \cf\ (\ref{eq:SpecFuncW}). 
This means that in the spectral 
function the spin part can be dealt with explicitly, using that 
 $\xi^\ast_{ss^{\phantom{\prime}}_z}\xi^{\phantom{\ast}}_{ss'_z} 
 = \delta_{s^{\phantom{\prime}}_z s'_z}$. 
So the factor 
 $\CG( s^{\phantom{z}}_a, s^z_a, s^{\phantom{z}}_b, s^z_b, S, S_z)$ 
will drop out together with the 
summation over $s^z_a$ and $s^z_b$, due to the condition that the total spin
 $S$ is the same in both two-body wave functions in each term of the spectral 
function (\ref{eq:SpecFuncW}).

\subsection{Coupled Correlated Two-Body Wave Functions}
In order to incorporate short-range correlations
the transformation to relative and \CM\
momenta is applied. Therefore the Moshinsky transformation
(\ref{eq:MTapply}) is combined with (\ref{eq:phiab}), in which the
transformation brackets are given by\cite{Ed57}
%
	\begin{eqnarray} 
		\inp<(l_a l_b) \lambda (s_a s_b)SJM |%
		     (l_a s_a) j_a (l_b s_b)j_b JM > 
	=
	&&
	\label{eq:jjlamS}
	\\
	&&
	\hspace{-1.5cm}
		\hat{\lambda} \hat{S} \hat{j}_a \hat{j}_b
		\negenJ(l_a, l_b, \lambda, s_a, s_b, S, j_a, j_b, J)
	\;,
	\nonumber
	\end{eqnarray} 
%
where $\hat{j}=\sqrt{2j+1}$. 
Combining (\ref{eq:jjlamS}) with the Moshinsky transformation 
(\ref{eq:MTapply}), the wave function becomes
%
	\begin{eqnarray}
		\Phi^{J_fM_f}_{ab} ( \rvec{p}_1,\rvec{p}_2 )
	&=&
		\inp< \rvec{p}_1 \rvec{p}_2| (a b) J_f M_f>
	\label{eq:ACtrafo1}
	\\
	&=&
		\sum_{Si}
		C^{ ab J_f}_{Si}
		\inp< \rvec{p}_{rel} \rvec{p}_{\CM} | i S  J_f M_f >
	\nonumber
	\;,
	\end{eqnarray}
%
where definitions for the used symbols are
%
	\begin{subeqnarray}
		ab
	&=&
		\{ n_a, l_a, j_a, n_b, l_b, j_b \}
	\;,
	\\
		i
	&=&
		\{n_i,l_i,N_i,L_i,\lambda_i\}
	\mbox{ and}
	\\
		C^{ ab J_f}_{Si}
	&=&
		\hat{\lambda_i} \hat{S} \hat{j}_a \hat{j}_b
		\negenJ(l_a, l_b, \lambda_i, s_a, s_b, S, j_a, j_b, J_f)
		\inp< n_i l_i N_i L_i \lambda_i| n_a l_a n_b l_b \lambda_i>
	\;.
	\slabel{eq:Cab}
	\end{subeqnarray}
%
The index $i$ has a finite range due to the energy conservation
condition of the Moshinsky brackets (\ref{eq:MTEC}).

At this stage SRC can be taken into account.
In the expression for the $(jj)$ coupled
wave function
%
	\begin{eqnarray}
		\inp< \rvec{p}_{rel} \rvec{p}_{\CM} | i S >
	&=&
		\sum_{ m, M, \mu, S_z }
		\CG( l_i, m, L_i, M, \lambda_i, \mu )
		\CG( \lambda_i, \mu, S, S_z, J_f, M_f )
	\nonumber \\
	&\times&
	\!\!
		R_{n_il_i}(|\rvec{p}_{rel}|)Y_{l_im}(\hat{\rvec{p}}_{rel})
		R_{N_iL_i}(|\rvec{p}_{\CM}|)Y_{L_iM}(\hat{\rvec{p}}_{\CM})
	\nonumber\\
	&=&
		\Phi_{iS}( \rvec{p}_{rel}, \rvec{p}_{\CM} )
	\nonumber 
	\end{eqnarray}
%
the relative radial wave function 
 $R_{n_il_i}(|\rvec{p}_{rel}|)$
can be replaced by the correlated relative wave
function, which is the Fourier-Bessel transform of $f(r)R_{nl}(r)$, with
 $f(r)$ one of the correlation functions of section~\ref{sec:cf}.
%
If, instead, one incorporates SRC by means of defect functions,
an extra angular momentum 
coupling should be applied as follows.

%\subsubsection{partial waves}
The defect functions calculated in ref.~\cite{MS93} are calculated
using partial wave angular momentum coupling.
Partial waves (notation $^{2S+1}l_{J'}$, \eg\ $^1S_0$) are coupled 
with the scheme:
%
	\begin{equation} 
  		\rvec{l} +  \rvec{S} = \rvec{J'} ;\;\; 
  		\rvec{J'} + \rvec{L} = \rvec{J}
	\end{equation} 
%
The required transformation bracket is\cite{Ed57} %(\cf\ FW-B.26):
%
	\begin{eqnarray} 
		\inp< (lL)\lambda S J| (lS)J' L J> 
	&=&
		(-)^{L+l-\lambda+L+J'-J}
		\inp< (Ll)\lambda S J| L (lS)J' J>  
	\nonumber \\
	&=&
		(-)^{L-\lambda+J'+S} \hat{\lambda} \hat{J'}
		\zesJ( L, l, \lambda, S, J, J')
		%      1  2  J'       3  4  J
	\label{eq:SixJ}
	\;,
	\end{eqnarray} 
%
hence the partial waves are written in terms of the known functions as
%
	\begin{eqnarray} 
	\lefteqn{
		\inp< \rvec{p}_{rel} \rvec{p}_{\CM}| n N (lS) J' L J_f M_f >
	=} \nonumber \\
	&&
		\sum_{ \lambda }
		(-)^{L-\lambda+J'+S} \hat{\lambda} \hat{J'}
		\zesJ( L, l, \lambda, S, J_f, J')
	\nonumber \\
	&\times&
		\sum_{ m, M_L, \mu, s^z_a, s^z_b, S_z }
		\CG( l, m, L, M_L, \lambda, \mu )
		\CG( s^{\phantom{z}}_a, s^z_a, s^{\phantom{z}}_b, 
		     s^z_b, S, S_z)
		\CG( \lambda, \mu, S, S_z, J_f, M_f )
	\nonumber \\
	&\times&
		R_{nl}(|\rvec{p}_{rel}|)Y_{lm}(\hat{\rvec{p}}_{rel})
		R_{NL}(|\rvec{p}_{\CM}|)Y_{LM_L}(\hat{\rvec{p}}_{\CM})
	\label{eq:PW}
	\;.
	\end{eqnarray} 
%
Using the inverse of the transformation (\ref{eq:SixJ}), and again noting
that the summation over $s^z_a$ and  $s^z_b$ can be discarded together
with the Clebsch-Gordan symbol since the total spin $S$ is the same
in the pair of conjugated two-body wave functions in the spectral function
(\ref{eq:SpecFuncW}), one may link the partial wave coupling to the earlier 
derived expression (\ref{eq:ACtrafo1}) 
%
	\begin{eqnarray}
	\lefteqn{%
		\inp< \rvec{p}_{rel} \rvec{p}_{\CM} | i S >
	=} &&
	\label{eq:PW2} \\
	&&
		\sum_{ J' }
		(-)^{L_i-\lambda_i+J'+S} \hat{\lambda_i} \hat{J'}
		\zesJ( L_i, l_i, \lambda_i, S, J, J')
		\inp< \rvec{p}_{rel} \rvec{p}_{\CM}| 
		      n_i N_i (l_iS) J' L_i J_f M_f >
	\nonumber
	\end{eqnarray}
%
The correlated wave function is constructed by adding the appropriate defect 
function to the partial waves in (\ref{eq:PW2}). 

\subsection{Total Spectral Function}
At this stage 
the total spectral function can be written out explicitly. Substitution
of the above mentioned angular momentum coupling, one obtains for the
strength of $S^-_J$ at energy $E^{n,A-2}_{J_f}$ 
%
	\begin{eqnarray}
	\lefteqn{
		S^-( \rvec{p}_1, \rvec{p}_2, \rvec{p}_{1'}, \rvec{p}_{2'}, n )
	=}
	\label{eq:CRSF} \\
	&&
		\sum_{abcd,M_f}
		\Psi^{\ast J_f M_f}_{cd}( \rvec{p}_{1'}, \rvec{p}_{2'} )
		X^{n\ast}_{cd J_f}
		X^{n}_{ab J_f}
		\Psi^{ J_f M_f}_{ab}( \rvec{p}_1, \rvec{p}_2 )
	\nonumber \\
	&&
	\!\!\!\!
	\!\!\!\!
	\!\!\!\!
	=
		\sum_{abcd}
		\sum_{SM_f}
		\sum_{ij}
		\inp< \rvec{p}_{rel'} \rvec{p}_{\CM'} | j S >
		C^{ cd J_f M_f }_{Sj}
		X^{n\ast}_{cd J_f}
		X^{n}_{ab J_f}
		C^{ ab J_f M_f }_{Si}
		\inp< i S | \rvec{p}_{rel} \rvec{p}_{\CM} >
	\nonumber \\
	&&
	\!\!\!\!
	\!\!\!\!
	\!\!\!\!
	=
		\sum_{abcd}
		\sum_{SM_f}
		\sum_{ij}
		\Psi^\ast_{jS}( \rvec{p}_{rel'},\rvec{p}_{\CM'})
		C^{ cd J_f M_f }_{Sj}
		X^{n\ast}_{cd J_f}
		X^{n}_{ab J_f}
		C^{ ab J_f M_f }_{Si}
		\Psi_{i S}( \rvec{p}_{rel}, \rvec{p}_{\CM} )
	\nonumber
	\end{eqnarray}
%
where the functions $\Psi$ are the correlated wave functions constructed
as described above.
