\begin{WTabstract}%
The importance of induced forces in the calculation of the nuclear 
charge-exchange response is studied. A feasible approximation to the 
Bethe-Salpeter equation for the polarization propagator is proposed and 
compared with various other recently proposed methods. The method is used 
to calculate of Gamow-Teller charge-exchange excitations in $^{48}$Ca; a 
nucleus with neutron excess for which particle-hole RPA methods predict 
negligible ($n$,$p$) strength in contrast to a `Dressed RPA' (DRPA) method. 
It is 
found\cite{GAD94} that the inclusion of an induced effective interaction by
means of the so-called `Dressed Extended' RPA (DERPA) method affects the DRPA 
results only slightly, which is an indication for the reliability of the latter.
This implies that a considerable fraction of the total strength resides in 
large tails up to more than $50$~MeV excitation energy. Therefore this seems 
to be the origin of undetected strength, rather than a reduction of the axial 
vector coupling in the nuclear medium, suggested by recent shell model studies.
\end{WTabstract}
