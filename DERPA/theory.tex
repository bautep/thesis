\section{Theory\label{sect:DERPAtheory}}
\subsection{ERPA\label{sect:theoryERPA}}
In order to be able to formulate an extension to the DRPA equation 
(\ref{eq:DRPA}) it is useful to repeat the derivation of ERPA\cite{BAD90}.
The particle-hole vertex, dictated by the Baym-Kadanoff relation
(\cf\ fig.~\ref{fig:SG2nd}) 
which was used to formulate ERPA, leads together with the 
second-order self-energy diagrams in the Dyson equation to a certain set of 
second-order contributions to the polarization propagator in the 
Bethe-Salpeter equation (\ref{eq:BS}). All these second order contributions 
are displayed in fig.~\ref{fig:DERPAdiagrams}.
%%%%%%%%%%%%
%
% figuur
%
\begin{figure}
\centerline{
\epsfig{figure=figures/erpa/RPA2ndorder.ps, height=.65in}\hspace{25pt}
\epsfig{figure=figures/erpa/selfscr1.ps,    height=.65in}\hspace{25pt}
\epsfig{figure=figures/erpa/selfscr2.ps,    height=.65in}\hspace{25pt}
\epsfig{figure=figures/erpa/screening.ps,   height=.65in}\hspace{25pt}
\epsfig{figure=figures/erpa/ladder.ps,      height=.65in}
}
\caption[]{All diagrams of second-order in $V$ (wiggly line) contribute to
the polarization propagator. The first diagram is the second order RPA diagram.
The next two are so-called self-screening diagrams that emerge from the
second order self-energy in the Dyson equation. The last two are contributions
due to the more elaborate form of the particle-hole vertex used in ERPA.
\label{fig:DERPAdiagrams}}
\end{figure}
%%%%%%%%%%%%%

In the approximation that all the lines in the diagrams of 
fig.~\ref{fig:DERPAdiagrams} are single-pole propagators (with eventually 
quasi-hole energies) these diagrams can all be written as
%
	\begin{equation}
		L^0_{\alpha\beta;\kappa\lambda}(\omega)
		\Gamma^{\mbox{\tiny ERPA}}_{\kappa\lambda;\mu\nu}(\omega)
		L^0_{\mu\nu;\gamma\delta}(\omega)
	\;.
	\end{equation}
%
The next step is now to insert this particle-hole vertex
into the ERPA equation (\ref{eq:ERPA}), such that the induced interaction is 
included to all orders.

\subsection{DERPA\label{sect:theoryDERPA}}
In modified form the above derivation can be used to obtain an extension of 
DRPA, which will be referred to as Dressed Extended RPA (DERPA).
The new ingredient in DERPA is that the 
single-particle propagators are dressed with a self-energy of at least 
second-order in $V$.
This implies that diagrams like the second and third diagram in
fig.~\ref{fig:DERPAdiagrams}  are supposed to have been  already included 
in $L^f$. 
Therefore only the screening and ladder diagrams 
(the fourth and fifth diagrams in fig.~\ref{fig:DERPAdiagrams}, respectively)  
will be incorporated in an effective interaction.

The way to derive ERPA was to write down the screening and ladder 
diagrams (fourth
and fifth diagram in fig.~\ref{fig:DERPAdiagrams}), and construct in this way 
the induced interaction.  If the same procedure is followed with dressed 
propagators, the
most awkward problem encountered is
that the expressions for the ladder and screening propagator do 
not factorize (note the implicit summing). 
This point is illustrated by a few characteristic terms in the screening 
propagator. 
%
%
% figuur
%
\begin{figure}
\centerline{
\epsfig{figure=figures/erpa/screeningDERPA.ps,   width=8cm}
}
\caption[]{The screening contribution (fifth diagram in 
fig.~\ref{fig:DERPAdiagrams}) $L^{\rm screening}_{12;34;J}(\omega)$ with 
fully dressed lines. Indices and variables correspond to formula 
(\ref{eq:screening}).
\label{fig:screeningERPA}}
\end{figure}
%
%
%
%(thesis W.~Hengeveld, p.36, thesis M.~Brand, appendix~C)
After angular momentum coupling
the following expression for the screening contribution with dressed lines
(fig.~\ref{fig:screeningERPA}) is obtained (repeated labels should be summed
over):
%
	\begin{eqnarray}
	\lefteqn{%
		L^{\rm screening}_{12;34;J}(\omega) = } 
	\label{eq:screening}
	\\
	&&
		\int {{\rm d} \omega_1 \over 2\pi i}
		{{\rm d} \omega_2 \over 2\pi i}
		{{\rm d} \omega_3 \over 2\pi i}
		\sum_{J'}
		%\sum_{5,6}
		\ME< 1' 3'^{-1};J' | V | 5' 6'^{-1};J' >
		\ME< 5  6^{-1};J'  | V | 2' 4'^{-1};J' >
	\nonumber \\
	&&
	\times
		(-1)^{j_2+j_3+J'+J+1}
		(2J'+1) \zesJ( j_1 , j_2 , J, j_4 , j_3 , J'  )
	\nonumber \\
	&&
	\times
		g_{11'}(\omega_2 + \half\omega )
		g_{2'2}(\omega_2 - \half\omega )
		g_{3'3}(\omega_1 + \half\omega )
		g_{44'}(\omega_1 - \half\omega )
	\nonumber \\
	&&
	\times
		g_{55'}(\omega_3 + \half(\omega_1 - \omega_2) )
		g_{6'6}(\omega_3 - \half(\omega_1 - \omega_2) )
	\nonumber 
	\;,
	\end{eqnarray}
%
where
$\ME< a b^{-1};J | V | c d^{-1};J >$ is the angular momentum coupled
G-matrix element. Details on angular momentum coupling are given in 
appendix~\ref{app:detail}.
The last line, together with the integrations, determines the structure 
as a function of $\omega$. When the integrations are
carried out using the diagonal approximation for $g$, where the diagonal
elements are of the  form eq.~(\ref{eq:g1s}),
one obtains factors of the form
%
	\begin{equation}
	%\hspace{-25pt}
	-\sfrac{\Sp{1} \Sh{2} \Sp{3} \Sh{4} \Sp{5} \Sh{6}}
	{ (
		- \omega
		+ \Ep{3} - \Eh{4} 
		-i \eta 
	)
	(
		- \omega
		- \Eh{2} + \Ep{3} + \Ep{5} - \Eh{6} 
		-i \eta 
	)
	(
		 \omega
		- \Ep{1} + \Eh{2} 
		+ i \eta 
	)
	}
	\label{eq:L2Upper} 
	\end{equation}
%
but also other ones, \eg\ of the form
%
	\begin{equation}
	\hspace{-1pt}
	\sfrac{\Sh{1} \Sh{2} \Sp{3} \Sh{4} \Sp{5} \Sh{6}}
	{(
		- \Eh{1} + \Ep{3} + \Ep{5} - \Eh{6}
		-i \eta 
	)
	(
		- \omega
		+ \Ep{3} - \Eh{4} 
		-i \eta 
	)
	(
		- \omega
		- \Eh{2} + \Ep{3} + \Ep{5} - \Eh{6} 
		-i \eta 
	)
	}
	\hspace{-8pt}
	\label{eq:L2Lower}
	\end{equation}
%
Note that neither of these formulas
are of the form $L^f \tilde{V} L^f$, for the implicit sums 
($i$, $j$, $5$ and $6$) extend
over the entire expression and not just over factors that can be identified
as terms of $L^f$. 

In the derivation of ERPA\cite{Br90}
terms of type (\ref{eq:L2Lower}) disappear by the single-pole 
approximation for the single-particle propagator and the condition that the 
external lines must be particle-hole pairs. Here, these terms will be dropped 
with the argument that the energy denominators are larger than those in 
(\ref{eq:L2Lower}). The latter term may still be used to formulate a RPA-type 
of equation albeit with a reformulation of the problem. For this purpose
 $L^f$ is no longer 
regarded as a diagonal matrix with as indices some coding 
which maps a pair \{$\alpha$, $\beta$\} to an index, such that $j_\alpha$ 
and $j_\beta$ can be coupled to the total angular momentum with on the diagonal
single-pole terms (like $L^0$ in (\ref{eq:ERPA})).
If the coding is reformulated such that $\{\alpha,i_\alpha,\beta,j_\beta\}$ 
denotes one index, then 
$L^f$ is a diagonal matrix with only zero or one term per entry (\cf\ 
(\ref{eq:drpa:Lf}))
and all the sums (except over $i_{5,6}$, $j_{5,6}$ and $5,6$) are performed 
in the 
matrix multiplication; so formula (\ref{eq:L2Upper}) factorizes. 
This factorization is achieved
at the cost of an increased dimension of the matrix that must be dealt with.
With this new coding an effective interaction matrix 
$\Gamma^{\mbox{\tiny DERPA}}$ can be defined,
and the DERPA equation becomes similar to the ERPA equation%
\cite{BAD90}:
%
	\begin{equation}
		L^{\mbox{\tiny DERPA}}_{ij}(\omega)
	=
		L^f_{ij}(\omega)
	+ 
		\sum_{kl}
		L^f_{ik}(\omega)
		\Gamma^{\mbox{\tiny DERPA}}_{kl}(\omega)
		L^{\mbox{\tiny DERPA}}_{lj}(\omega)
	\;.
	\label{eq:DERPA}
	\end{equation}
%
%%
It should be noted that the matrices in this equation are very much larger 
than for (E)RPA or DRPA. If a single-particle propagator  
(\ref{eq:g1s})
contains about $n$
poles, the indices $i$, $j$ \etc\ in (\ref{eq:DERPA}) 
represent a dimension of about $n^2$ values. 
So without further approximations this method will 
quickly exceed available computer possibilities. Therefore it would be useful to
 have an approximate expression for the single-particle propagators with a 
much smaller number of poles, for instance as is obtained with the BAGEL 
method of ref.\cite{MS93}, where the Lanczos procedure is used to calculate a 
few representative poles of the single-particle propagator. In the following we shall adopt a rather 
{\em ad hoc} procedure where the single-particle propagators are represented 
by an expression with only three poles. 
This already gives rise to nine times larger matrices than in (D)RPA. 

An additional problem that arises in the DERPA method (\ref{eq:DERPA}) with 
an artificially reduced set of poles of the single-particle propagators 
manifests itself when the most rigorous approximation is made. 
If one takes a single-pole approximation for the single-particle propagators 
one obtains just the screening and ladder diagrams for the effective 
interaction in the ERPA formalism of ref.~\cite{BAD90}. However, the 
so-called self-screening diagrams, which were also found to be essential in 
ERPA are then not included in (\ref{eq:DERPA}), because they then occur 
neither in $L^f$ nor in $\Gamma$. So this is an indication that approximation 
of (\ref{eq:DERPA}) with a too small number of poles for the single-particle 
propagator cannot be justified. One should at least check in all cases the 
degree of violation of conservation laws, \ie\ sum rules.
