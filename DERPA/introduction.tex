\section{Introduction}
For a long time there has been considerable interest in Gamow-Teller (GT) 
strength distributions. First, there was the suggestion by Ikeda\cite{IFF63} 
and Ejiri\cite{EIF68} that the weakness of low-energy GT-transitions should 
be related to approximate symmetries of the nuclear Hamiltonian and thereby to 
GT-resonances at higher energies. When indeed these resonances were discovered 
in ($p$,$n$) and ($^{3}$He,$t$) reactions%
\cite{GLH80,GGG80,GRT81,HGB81,ACB85,CAB86}
it turned out that the total observed strength was less than expected and 
speculations came up that this was due to the coupling to an even higher 
resonance, the $\Delta_{33}$ resonance of the nucleons%
\cite{BM81,BCT81,OKS82}. Then many theoretical discussions followed%
\cite{DNS90,Ga90,Rij93,RGBA93}
with conclusions tending in the direction that the influence of the $\Delta$ 
at low energies is small, if present at all, and that the GT-strength is 
partly present in what was first considered as experimental background and 
also partly spread over a much larger than the experimentally covered energy 
range\cite{DNS90,Rij93}. It has also been shown that the total strength 
increases by a $25$\% or more when the partial occupation of `occupied' shells 
is taken into account\cite{ALV85,NDW88,Rij93}, but this increase mainly goes 
to higher excitation energies, outside the experimentally analyzed 
region\cite{Rij93}. The latter conclusion was reached, however, by a method 
which did not obey the Baym-Kadanoff condition (\ref{eq:BK}) to guarantee 
conservation laws, although it was checked that sum rules were quite well 
satisfied.
In the mean time there were suggestions\cite{MKT94} that the calculated 
GT-strength depends strongly on the schemes that are adopted and that the 
theoretical amount of strength, both ($p$,$n$) and ($n$,$p$), is quite 
uncertain. In view of these uncertainties it seems worthwhile to extend the 
Dressed RPA (DRPA) calculations of ref.~\cite{Rij93} in such a way that the 
self-energy and the effective particle-hole interaction are treated on more 
equal footing and thereby more in agreement with the Baym-Kadanoff 
relation (\ref{eq:BK}) although this relation is not exactly satisfied.

The possible paths to extensions%
\cite{BAD90,TSA88a,Saw62,BH82,YDG83,DNS90,HDA86}
 are dictated by the way of deriving the 
RPA-equation (\ref{eq:RPA}); \eg\ equations of motion method\cite{RS80} or
polarization propagator\cite{FW71}. In the line of the derivation of RPA via
the polarization propagator (as given in chapter~\ref{chap:GREEN}), Extended
RPA (ERPA) was proposed\cite{BAD90}, inspired by the fact that a mean-field
description of the nuclear system fails to describe the single-particle 
properties like the fragmentation of the one-hole spectral function 
(\ref{eq:OHSF}) that has been observed in one-proton removal experiments \eep\
\cite{HBJ88,Qu88a,Kr90,Ste91}.
The fragmentation of the one-hole spectral function requires a description 
with at least a second order self-energy (fig.~\ref{fig:SG2nd}). According to 
the Baym-Kadanoff relation (\ref{eq:BK}), a more elaborate particle-hole
interaction $\Gamma$ (fig.~\ref{fig:SG2nd}) should then be incorporated.

In earlier studies of the Gamow-Teller response on $^{48}$Ca, a double magic 
nucleus with neutron excess, it appeared that RPA as well as ERPA is not 
able to account for the experimentally found ($n$,$p$) strength. This strength 
is exactly zero in MFA. The Pauli principle prohibits the change of a 
proton into a neutron with the same spatial and spin quantum numbers. 
The partial 
depletion of lower (neutron) shells would make these transitions possible and 
the simplest scheme that allows for these effects is the Dressed 
RPA\cite{RGBA93} (\ref{eq:DRPA}).
In ref.~\cite{RGBA93} the so-called Dressed Independent Particle Approximation
(DIPA) was introduced, consisting of the approximation of no residual 
interaction within the DRPA equation (\ref{eq:DRPA}) such that the 
propagating particles interact with the medium (via the Dyson equation) but 
not with each other.
The similarity of the DIPA and DRPA results for the Gamow-Teller 
($n$,$p$)-strength leads already to the preliminary conclusion\cite{RGBA93} 
that for the description of response the first term of the Bethe-Salpeter 
equation is the most important. 
In recent shell model calculations\cite{CPS95} an experimental (rather crude) 
quench factor of $0.7$
was introduced in the calculation of Gamow-Teller ($p$,$n$) response, due 
to depletion of the orbits that are involved in this transition.
Nevertheless the DRPA particle-hole vertex (which was
taken to be just the residual interaction) is not constructed according to the
Baym-Kadanoff relation (\ref{eq:BK}). 

In this chapter a feasible scheme is given to go beyond the DRPA using the
ERPA as example. This scheme will be given in section~\ref{sect:DERPAtheory}, 
while in section~\ref{sect:DERPAresults} results of the Gamow-Teller response
on $^{48}$Ca will be given.
