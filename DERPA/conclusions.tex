\section{Conclusions}
In summary, the conclusion seems 
justified that already in DRPA the most important coherence effects of the 
interaction $\Gamma$ are included and that therefore this method, in which 
the fully fragmented $L^f$ can be handled, is quite suitable to predict 
especially the tails of excitation strengths distributions. For detailed 
descriptions of the ($p$,$n$) peak region the ERPA should be considered as 
a viable alternative. A comparison of DERPA with ERPA cannot easily be made in 
a transparent manner because in ERPA the effects of self-energy and induced 
interaction, figs~\ref{fig:SGHF} and \ref{fig:SG2nd}, are intertwined into an 
effective interaction in which they partly compensate each other. In DERPA 
this compensating effect finds expression in a weakening of the induced 
forces, fig.~\ref{fig:SG2nd}, as compared to the same diagrams calculated with 
single-pole expressions for the single-particle propagators.

In shell model calculations\cite{CZPM94,CPS95,LDR95} 
a quench factor has been introduced
to explain the observed strength. 
This factor is then associated with suggestions\cite{LDR95} that the axial 
coupling constant $g_A$ is smaller for nucleons in the nuclear medium than for 
free nucleons, $g_A^{eff}=1$ instead of $1.26$. The present calculations 
present no reason to invoke such a quenching; the reduction of strength in the 
experimentally analyzed region originates from the depletion of the valence 
orbits involved in the GT-transitions, while the partial occupation of remote 
orbits leads to a substantial strength at higher energies. This latter feature 
cannot be reproduced by shell model calculations of the type of 
refs.~\cite{CZPM94,CPS95,LDR95}. Therefore we believe that their conclusions
that `essentially all strength occurs below $15$~MeV' and the 
indication for a reduced $g_A$ based on their work are wrong.

%In DRPA this factor is already present, for
%it originates from the depletion of the orbits involved in the Gamow-Teller
%transition. A further feature of DRPA is that also ($n$,$p$) strength is 
%predicted to be sizable\cite{RGBA93}, which is more in line with measured
%($n$,$p$) strength in $^{40}$Ca\cite{PRU92} 
%(measurements for $^{48}$Ca are still unavailable). 
%This ($n$,$p$) strength will play a role in the discussion about
%Gamow-Teller strength missing according to the Ikeda sum rule\cite{RGBA93}; 
%more ($p$,$n$) strength is missing in that case.
