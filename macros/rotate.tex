%
%   These macros allow you to rotate or flip a \TeX\ box.  Very useful for
%   sideways tables or upsidedown answers.
%
%   To use, create a box containing the information you want to rotate.
%   (An hbox or vbox will do.)  Now call \rotr\boxnum to rotate the
%   material and create a new box with the appropriate (flipped) dimensions.
%   \rotr rotates right, \rotl rotates left, \rotu turns upside down, and
%   \rotf flips.  These boxes may contain other rotated boxes.
%
\newbox\rotbox\newdimen\rotdimen
\def\escspec#1{\special{"@endspecial #1 @beginspecial @setspecial}}%
\def\rotfinish{\escspec{currentpoint grestore moveto}}%
%
%   First, the rotation right.
%
\def\rotr#1{\rotdimen=\ht#1\advance\rotdimen by\dp#1%
   \hbox to\rotdimen{\hskip\ht#1\vbox to\wd#1{\escspec{gsave
   currentpoint currentpoint translate 90 rotate neg exch neg exch
   translate}\box#1\vss}\hss}\rotfinish}
%
%   Next, the rotation left.
%
\def\rotl#1{\rotdimen=\ht#1\advance\rotdimen by\dp#1%
   \hbox to\rotdimen{\vbox to\wd#1{\vskip\wd#1\escspec{gsave
   currentpoint currentpoint translate 270 rotate neg exch neg exch
   translate}\box#1\vss}\hss}\rotfinish}%
%
%   Upside down is simple.
%
\def\rotu#1{\rotdimen=\ht#1\advance\rotdimen by\dp#1%
   \hbox to\wd#1{\hskip\wd#1\vbox to\rotdimen{\vskip\rotdimen
   \escspec{gsave currentpoint currentpoint translate 180 rotate neg exch
   neg exch translate}\box#1\vss}\hss}\rotfinish}%
%
%   And flipped end for end is pretty easy too.
%
\def\rotf#1{\hbox to\wd#1{\hskip\wd#1\vbox{\escspec{gsave currentpoint
   currentpoint translate -1 1 scale neg exch neg exch translate}\box#1}%
   \hss}\rotfinish}%
%
%
%
\def\rotleft#1{\setbox\rotbox=\hbox{#1}\rotl\rotbox}%
%
\def\rotright#1{\setbox\rotbox=\hbox{#1}\rotr\rotbox}%
%
\def\rotupside#1{\setbox\rotbox=\hbox{#1}\rotu\rotbox}%
%
\def\rotflip#1{\setbox\rotbox=\hbox{#1}\rotf\rotbox}%

