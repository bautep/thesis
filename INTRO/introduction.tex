%\section{Introduction (and Summary)}
This thesis deals with the description of atomic nuclei as an assembly of 
protons and neutrons, which are considered as inert objects. 
The nuclear system is studied experimentally by stirring it with a probing 
particle (\eg\ a proton or an electron). The aim of a theoretical description is
to describe and understand the output measured in such an experiment. 
Therefore a many-fermion problem has to be solved.
This has been studied in various approximations for several decades.
Besides the use of nuclear models%
\cite{DeT63,EG72,dSF74,BM75,BG77,RS80}, 
there have always been attempts to describe nuclear 
properties as a many-body problem, starting with a realistic nucleon-nucleon 
($NN$) interaction that reproduces the scattering phase shifts 
\cite{KB66,KB68,BK73,MMB90,FM94}. 
The latter type of calculations has become feasible due to the 
rapidly increasing computer performance. One may distinguish two kinds of 
development. 

In the first approach one constructs many-body wave functions within a limited 
(shell) model space. For this purpose it is then necessary to construct from 
the $NN$-interaction an equivalent effective model space interaction as well 
as effective operators for observables. In principle such a procedure may be 
well-defined\cite{Br67a,BJB71,MPK85} 
and some work along this line is still going on\cite{SOE94}. 
An essential drawback of the method is that, even in a relatively
small model space of only a few major oscillator shells for protons as well as
neutrons, the dimensions of the matrices that one has to deal with explode 
with the growth of 
the size of the model space and soon become of order $10^{12}$ or higher,
which is a reason to resort again to phenomenological models\cite{Ta92}.

In the second kind of approach, one partly circumvents these huge dimensions by 
dealing directly with matrix elements of observables, for which one studies 
equations of motion. 
These observables depend on the reaction studied, \eg\ one-nucleon removal is 
described by the solution of the Dyson equation while two-nucleon removal and 
electro-magnetic response are described by solutions of a Bethe-Salpeter 
equation.
In the framework of Green's functions techniques, 
with its extensive applications in field theory and quantum statistical 
mechanics\cite{AGD63,FW71,KE88},
these equations of motion describe the Green's functions with summations of 
infinite sets of Feynman diagrams.
Approximations can be made by considering the correspondence of these diagrams
with physical processes.
The diagrams which are believed to be the most relevant for the 
properties under investigation,
are then taken into account in the summation.
 In the early days various building blocks of 
this theory were represented by phenomenological parameters\cite{Mi67} as 
their calculation from the fundamental interactions was considered to be 
hopelessly complicated. In recent years, however, due to the enormous growth of 
computational tools, {\em ab-initio} calculations are becoming feasible. 

Recently, self-consistent calculations of the nuclear spectral functions in 
nuclear matter\cite{RPD89,VDP93} 
and in finite nuclei\cite{MD94} have been reported 
with a Reid soft-core or Bonn nucleon-nucleon ($NN$)
 potential. Also a two-nucleon spectral 
function in nuclear matter has been calculated recently\cite{GDPR95}. In 
these nuclear matter studies the focus is on the short-range correlations, 
induced by the rather hard core of  the $NN$-potential. 
Because the extension of this hard core ($r_c \sim 0.4-0.5$~fm) 
is much smaller than the average inter-particle distance of 
$1.6$~fm,  the effect of the hard core is assumed to be taken into account when
all the so-called `ladder' diagrams (describing multiple binary collisions) 
are summed. However, the work of ref.~\cite{GDPR95} supersedes 
this by treating the scattering of `dressed nucleons'.
The nuclear system has this short-range effect in common with 
quantum liquids, in contrast 
to, for instance, 
many-body systems such as electron clouds of atoms and molecules 
and the electron 
gas in solids. 

In this thesis the effect of short-range correlations (SRC) on the
nuclear many-body system is treated in two stages. The heavy 
binary collisions in a nuclear environment are solved first, 
leading to an effective interaction (called Brueckner G-matrix)
and modified basis wave functions. Next, the nuclear 
motion associated with the finite size of the nucleus, 
notably shell structure and low-energy 
excitations, are calculated with this effective interaction. 
In the hard collisions intermediate states 
with high (relative) momenta are essential. The space of these high-momentum 
wave functions is largely orthogonal to the large but limited shell model 
space within which the longer-ranged low-energy dynamics is 
described.
Also the finite range of the force, which makes the treatment of screening to 
all orders unnecessary, is different from the case of the electron 
gas\cite{KE88} as an effective interaction. 

The G-matrix interaction used in 
chapter~\ref{chap:DERPA}, had been deduced in nuclear matter calculations and 
therefore 
the orthogonality of the spaces dealing with the different aspects is at best 
approximate. In the chapters~\ref{chap:SPECFAC} and \ref{chap:PPO} the 
applied G-matrix 
had been constructed in a space explicitly orthogonal to the shell model space.
The nuclear properties studied in this thesis are 
restricted to low-energy phenomena, \ie\ the structure is calculated
within a large but limited shell model space with a G-matrix as an effective 
interaction.
In the chapters \ref{chap:ppknock} and \ref{chap:PPO} the 
high-momentum part of the two-nucleon spectral functions is investigated.
Therefore, both the low-energy structure and the short-range correlations 
are put together. The input provided by calculations of 
M\"uther\cite{MS93a} had been essential here.

The composition of this thesis is as follows. In chapter~\ref{chap:GREEN}
a survey is given 
of recent applications of Green's functions methods to the description of 
low-energy nuclear properties and of which most of the other chapters may be 
considered as extensions in some respects. Approximations exploring the 
physical processes that have the most significant influence on 
the relevant propagators are discussed. In chapter~\ref{chap:DERPA}
an attempt is made to combine 
the merits of recent extensions \cite{BAD90,RGBA93} of the 
Random Phase Approximation 
(RPA) for the charge-exchange  response. The framework in which these 
extensions are formulated is the
 Bethe-Salpeter equation (BSE) for the polarization propagator. 
The conclusions of ref.~\cite{RGBA93}, \viz\ that the Gamow-Teller response 
in a ($n$,$p$) reaction on a magic nucleus with neutron excess is 
quite well described 
as soon as the partial filling of shells is taken into account with 
a `Dressed RPA' procedure, is confirmed. This conclusion  holds 
because it is found that the response strength 
is not essentially modified when certain components induced 
by medium polarization are included in the effective interaction.

In chapter~\ref{chap:SPECFAC} attention is focussed on the fact that the 
G-matrix  which is obtained by solving the Bethe-Goldstone equation 
(BGE)\cite{MS93a}, is energy-dependent. Therefore, if it is applied in the 
calculation of the mean field and of the spectral functions at low missing 
energies, already the Hartree-Fock term in the self-energy becomes 
energy-dependent. 
This energy-dependence then simulates a dispersive effect, \viz\ 
the scattering of a nucleon into orbits outside of 
the model space due to hard collisions. This is just the type of scattering 
dealt with in the BGE.
By treating the energy-dependence of the G-matrix explicitly this effect 
of short-range correlations is re-introduced into the model space calculation 
of spectroscopic factors. As expected this yields a depletion of spectral 
strength as compared with calculations in which the G-matrix is treated as a 
static effective interaction and in which dispersive and correlation effects 
arise only by diagrams of second order in the G-matrix interaction.
The results obtained are indeed in line with direct calculations of depletion 
of orbits by short-range correlations\cite{MPD95}. The fact that there is 
still a considerable discrepancy between the calculated spectroscopic factors 
and those deduced from a careful analysis of the data \cite{Leu94} calls for 
further theoretical investigation, however.

Chapters \ref{chap:ppknock} to~\ref{chap:CROSS} deal with the question whether 
short-range correlations between two nucleons are directly visible in the 
knock-out of two nucleons with high relative momenta. 
For that purpose a detailed expression for the two-hole spectral function is 
derived in chapter~\ref{chap:ppknock}. The SRC, dealt with in the BGE and 
incorporated in the G-matrix, yield also high-momentum components in the 
(relative two-body) wave functions. These high-momentum components are 
described by the defect functions calculated also from a BGE. The calculated 
spectral function is proportional to the $W^{00}$ element of the hadronic 
tensor in plane wave approximation and therefore comparable with the 
longitudinal structure function in the \eepp\ cross-section.
In chapter~\ref{chap:PPO} the spectral function derived in 
chapter~\ref{chap:ppknock} is calculated for $^{16}$O. 
The G-matrix and the defect functions were calculated by M\"uther.
The short-range effects for different realistic $NN$-potentials are compared
and the spectral function calculated with the Reid soft-core potential is 
roughly a factor two larger than when the Bonn(A) potential is used for the
calculation of short-range effects. In chapter \ref{chap:CROSS} the 
longitudinal part of the \eepp\ cross-section in plane wave approximation is 
calculated for a restricted set  of kinematics. 


Finally, conclusions are drawn in chapter~\ref{chap:OUTLOOK}.
