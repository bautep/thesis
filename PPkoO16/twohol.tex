\section{Two-Hole Energy Spectra\label{sec:twohol}}
%\subsection{Two-proton hole states}
In the spectrum\cite{Ajz1315} of the two-proton hole nucleus $^{14}$C,
corresponding
to $^{16}$O, one finds states with energies and 
angular momentum/parity ($J^\pi$) quantum numbers which are suggestive for 
their interpretation as two-proton holes, \eg\ in the $1p$ shells, 
but there are also states that must be ascribed to other mechanisms.
For instance, the 
 $1^-$ state at $6.094$ MeV, $0^+$ at $6.589$ and $3^-$ at $6.728$ MeV,
are clearly reminiscent of similar states in $^{16}$O at $7.117$, $6.049$ and
 $6.130$ MeV\cite{Ajz1617}, respectively. So, these states may be interpreted 
as excitations
in $^{14}$C of the $^{16}$O core. These states, which do not allow a simple 
two-hole interpretation, are expected to be hardly populated in two-proton 
knockout from $^{16}$O. This is confirmed by a two-proton pick-up 
experiment $^{16}$O($n$,$^3$He) in which only the $^{14}$C ground state and 
$2^+$ at about $7$ MeV were clearly visible\cite{Ajz1315,Ajz1617}.

Much clearer, though indirect, information is available from the isospin 
 mirror reaction $^{16}$O($p$,$t$)$^{14}$O\cite{FHC71,Mac74}. In these 
experiments the
strongly populated states of $^{14}$O are the $0^+$ ground state, 
$2^+$ state
at $6.59$ MeV and $2^+$ at $7.78$ MeV. No $1^+$ state, 
composed of a $p\half$ and
 $p\threehalf$ hole, is seen in these reactions because in this 
configuration the 
relative wave function must correspond to $^3P_1$, 
whereas in the picked-up triton one has 
predominantly a $^1S_0$ configuration. 
So, as a result, there is no experimental 
information about this state, but we consider the $1^+$ state at 
$11.31$ MeV in $^{14}$C as a 
likely candidate for a $1^+$ two-proton $(1p\half; 1p\threehalf)^{-1}$ 
configuration. Other two-proton hole states in $^{14}$C are expected at 
higher excitation energies. The energies and main amplitudes that were obtained
in a `discrete' DRPA calculation are given in table~\ref{tab:CveertienAmp}.
Note that only the first few states in this table will be observable as discrete
states. The strength of the negative parity states is spread over a large 
energy region \cf\ figs.~\ref{fig:S1S0} and~\ref{fig:S3P}.
%%%%%%%%%%%%%%
%
% remark: calculated energies are with respect to the 0+ ground state 
% @ -23.75
% data can be found in directory (if existing:) 
% /home/hardy/theorie/geurts/wybo/programs/thesis/S_ab
% in files: drpa_pp_0+ etc.
%
\begin{table}
\centerline{
\begin{tabular}{|c|c|c|c|}
\hline 
 $J^\pi$ & $E^\ast_{\rm calc}$ &  Main amplitudes &  $E^\ast_{\rm exp.}$ \\
\hline 
%
%
	$0^+$	& 0	% -23.75		%%% line 30
		&	$0.77
			*(p\half)^{-2}		%%% #5
		$ , 
			$-0.11
			*(p\threehalf)^{-2}$	%%% #3
		& 0 
\\
	$2^+$   & 7.80	% -31.55		%%% line 50
		&	$-0.75
			*(p\half p\threehalf)^{-1}	%%% #4
		$ , 
			$0.12
			*(p\threehalf)^{-2}$		%%% #3
		&   $7.01$/$8.32$
\\
	$1^+$   & 6.93 	% -30.68		%%% 
		&	$-0.77
			*(p\half p\threehalf)^{-1}$ 	%%% #3
		& $11.31$
\\
	$0^+$   & 14.64	% -38.39		%%% 
		&	$0.08
			*(p\half)^{-2}		%%% #5
		$ , 	$0.63
			*(p\threehalf)^{-2}$ 	%%% #3
		&  9.75
\\
	$2^+$   
		& 13.76	% -37.51		
		&	$-0.10
			*(p\half p\threehalf)^{-1} 	%%% #4
		$ ,	$-0.75
			*(p\threehalf)^{-2}$ 		%%% #3
		&   
\\
\hline 
\end{tabular}
}
\caption[]{
States of $^{14}$C that are expected to be strongly populated in 
 $^{16}$O\eepp. The calculation is done with the `discrete' DRPA method
where the single-particle Green's functions of fig.~\ref{fig:sfO} are used.
The interaction was taken to be a G-matrix derived from the Bonn potential.
The negative
parity states are highly fragmented, so no sensible amplitudes could be 
extracted \cf\ fig.~\ref{fig:S1S0} and~\ref{fig:S3P}.
The experimental energies are taken from ref.~\cite{Ajz1315}
\label{tab:CveertienAmp}}
\end{table}
%%%%%%%%%%%%%%
