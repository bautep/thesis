\section{Introduction}
For a study of the two-nucleon removal spectral function (\ref{eq:CRSF})
the nucleus $^{16}$O is chosen here as the most suitable one. It may be 
expected that just a few shells, $1p$ and $1s$, yield the main contributions 
in 
the knock-out process and a small set of dominant two-hole configurations can 
be built with these shells. These dominant configurations are supposed to 
correspond to a clearly discernible set of final states in $^{14}$C. 
The energy resolution of a few MeV, which can be realized in some modern 
experimental setups, may be sufficient to separate cross-sections 
corresponding to the knock-out of pairs with well-defined angular momentum and
parity belonging to the dominant two-hole configurations.
The prospects for the separation discrete dominant states are less 
favorable for 
nuclei heavier than $^{16}$O, because of the larger number of orbits and the 
corresponding higher level density. In lighter nuclei, like $^{12}$C, 
low-energy correlations may 
complicate the interpretation though there are indications that knock-out 
from the $1p$ and $1s$ shells can be distinguished\cite{Kes93}.

In the present study our main purpose is to investigate the 
sensitivity of the
spectral function at high momenta of the knocked-out particles 
to different realistic $NN$-potentials. This is important 
because one hopes\cite{Prop} to learn about 
medium effects related to short-range correlations and 
thereby about the short-range part of the $NN$-potentials from two-nucleon 
knock-out at high momenta.
The potentials considered here are the Reid potential\cite{Re68} and the 
Bonn-A/C
potentials\cite{Ma89}. The latter two have a softer core than the former.

In section~\ref{sec:PPcompset} the computational ingredients will be given. In 
section~\ref{sec:defect} the relevant defect wave functions will be 
discussed.
In section~\ref{sec:twohol} the two-hole spectrum as obtained with the DRPA 
calculation will be given. The spectral function is studied in 
section~\ref{sec:specfunc}, while in section~\ref{sec:PPconclusions} 
conclusions are drawn.
