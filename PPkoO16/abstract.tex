\begin{WTabstract}%
The procedure outlined in chapter~\ref{chap:ppknock} is applied to  the 
calculation of the two-proton hole spectral function of ${}^{16}$O. The high
momentum parts of the relative wave function used in this procedure are
calculated using the Reid Soft Core (RSC) potential as well as the Bonn-A/C
potential. The RSC potential yields a 
spectral 
function in the momentum range of $3$--$5$~fm${}^{-1}$ for the 
ground state to ground state transition 
which is a factor two to three larger than the spectral function calculated
with the Bonn potentials. In a  separate calculation correlation 
functions, calculated in nuclear matter,  are  used to describe the 
high-momentum part of the wave function. These correlation functions  
yield more distinct
short-range correlations (SRC) at lower relative momenta.
The contribution of the 
relative S-wave is found to be the largest part 
of the two-hole spectral function at high momenta and therefore easiest 
accessible to experimental investigation. The contribution of the relative 
P-wave is smaller and therefore more difficult to identify.
\end{WTabstract}
