\section{Conclusions\label{sec:PPconclusions}}
The two-nucleon removal spectral function of $^{16}$O is studied for high 
momenta of the nucleons. These high momenta are introduced into shell model 
(relative) wave functions by means of the addition of defect functions 
calculated with a Bethe Goldstone equation\cite{MS93a} and also by means of 
correlation functions obtained in variational calculations\cite{Cl81}. Within 
the shell model space, of four major shells, the long-range correlations are 
treated with the Dressed RPA (DRPA). In this way
the two-nucleon removal amplitudes 
leading to the final states in $^{14}$C are computed with the G-matrix from 
the Bonn-C potential as a shell model (DRPA) interaction.
These amplitudes are used together with the defect or correlations functions 
to investigate the effect of short-range correlations on the two-nucleon 
removal spectral function. It is found that at high momenta
($2$--$5$~fm$^{-1}$) this spectral function is roughly a factor two larger 
when calculated
with the defect functions of the Reid potential than with those of the Bonn-A 
or Bonn-C potential. Distinct shapes and much larger differences for the 
spectral functions are obtained with the correlation functions, which were 
deduced\cite{Cl81} from old $NN$-potentials (Kallio-Kolltveit or Omhura-Morita-
Yamada).

Since the ground state of $^{14}$C is well-separated from the excited states, 
a missing-energy resolution of $4$~MeV in \eepp\ experiments is sufficient to 
study its spectral function. The largest cross-sections must be expected for 
the $2^+$ states at about $7$--$8$~MeV, however. This is indicated by the 
$^1S_0$ pair removal spectral function of fig.~\ref{fig:S1S0}. To separate 
these $2^+$ states a missing-energy resolution better than $1$~MeV is required. 
A study of short-range correlations in the $^3P$ states will be much more 
difficult, because of their small defect functions at high relative momenta. 
The best possibility for its experimental investigation might be to filter out 
the center-of-mass $L=0$ part (\cf\ fig.\ref{fig:S3P}) of the cross-section 
and to look at a missing energy of $40$--$50$~MeV.
