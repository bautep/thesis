\chapter{Details of the Calculation of Gamow-Teller Response\label{app:detail}}
\markboth{Details of the Calculation of \ldots}{}
%%%
\section{Angular Momentum Coupling}
If the one-body operator in (\ref{eq:defS}) 
is a spherical tensor operator\cite{Ed57}
of rank $\lambda$, it can be written as
%
	\begin{equation}
		\opO^\lambda_\mu 
	=
		\frac{1}{\sqrt{2\lambda+1}}
		\RME< a|\opOhatless^\lambda|b>
		(\Oc{a} \Oa{\tilde{b}})_{\lambda\mu}
	\label{eq:app1}
	\;,
	\end{equation}
%
where $\RME< a |\opOhatless^\lambda|b >$ is the reduced matrix element,
 and
%
	\begin{equation}
		(\Oc{a} \Oa{\tilde{b}})_{\lambda\mu}
	=
		\sum_{m_a m_b}
		\CG( j_a, m_a, j_b, m_b, \lambda, \mu )
		(-1)^{j_b - m_b}
		\Oc{a} \Oa{-b}
	\label{eq:app2}
	\end{equation}
%
in which the notation $a=\{n_a,l_a,j_a,m_a\}$ and $-b=\{n_b,l_b,j_b,-m_b\}$ are 
used. The symbols $\CG( j_a, m_a, j_b, m_b, \lambda, \mu )$ are the 
Clebsch-Gordan coefficients.
By introduction angular momentum coupling, (\ref{eq:defS}) becomes
%
	\begin{equation}
		S_{\opO^\lambda}(\omega)
	=
		-\frac{1}{\pi} {\rm Im} 
	\; 
		\sum_{\alpha\beta\gamma\delta}
		\RME<\alpha |\opOhatless^\lambda |\beta >^*
	\;
		\Pi_{\alpha\beta;\gamma\delta;\lambda}(\omega)
	\;
		\RME<\gamma |\opOhatless^\lambda |\delta >
	\;,
	\label{eq:app3}
	\end{equation}
%
where $\Pi$ is the coupled form of the polarization propagator $L$ 
(\ref{eq:DefLLeh}):
%
	\begin{eqnarray}
		\Pi_{\alpha\beta;\gamma\delta;\lambda}(\omega)
	&=&
		\sum_{n\not=0}
		\sum_{\mu=-\lambda}^{\lambda}
		\frac{1}{2\lambda+1}
		\left[
			\frac{
			\ME<\Psi_0|
				(\Oc{\beta} \Oa{\tilde{\alpha}})_{\lambda\mu}
			|\Psi_n >
			\ME<\Psi_n|
				(\Oc{\gamma} \Oa{\tilde{\delta}})_{\lambda\mu}
			|\Psi_0 >
    			}{
			\omega-E^n+i\eta
    			}
		\right] 
	\nonumber \\
	&&
	\hspace{-2cm}
	-
		\sum_{m\not=0}
		\sum_{\mu=-\lambda}^{\lambda}
		\frac{1}{2\lambda+1}
		\left[
			\frac{
			\ME<\Psi_0|
				(\Oc{\gamma} \Oa{\tilde{\delta}})_{\lambda\mu}
			|\Psi_m>
			\ME<\Psi_m|
				(\Oc{\beta} \Oa{\tilde{\alpha}})_{\lambda\mu}
			|\Psi_0>
			}{
			\omega+E^m-i\eta
			}
		\right]
	\;.
	\label{eq:app4}
	\end{eqnarray}
%
Using the fact that the summations are not dependent on $\mu$  
((\ref{eq:app2}) also defines a tensor operator), one can remove 
$\mu$ together with the factor $1/(2\lambda+1)$.

Angular momentum coupling implies that the following objects are used in 
formulas (instead of the uncoupled $V_{\alpha\beta\gamma\delta}$)\cite{ABBS88}
%
	\begin{eqnarray}
		\ME< a b;J | V | c d; J >
	&=& 
		G(abcdJ)
	\hspace{-3cm}
	\\
	&&
	\hspace{-3cm}
	= 
		\sum_{m_a m_b m_c m_d}
		\CG( j_a, m_a, j_b, m_b, J, m_a+m_b)
		\CG( j_c, m_c, j_d, m_d, J, m_c+m_d )
		V_{\alpha\beta\gamma\delta}
	\nonumber
	\end{eqnarray}
%
and
%
	\begin{eqnarray}
		\ME< a b^{-1};J | V | c d^{-1};J >
	&=&
		F(abcdJ)
	\\
	&&
	\hspace{-4cm}
	=
		-\sum_{J'} (2 J'+1)
		(-)^{j_a+j_b+j_c+j_d}
		\zesJ( j_a, j_b, J, j_d, j_c, J' )
		\ME< d a; J' | V | b c; J' >
	\nonumber
	\end{eqnarray}
%
%%%%%%%
% equation to be solved:
\section{Solution to the Coupled Equation}
Solving the equation (\ref{eq:DRPA}), one has to have in mind that
the real object of our interest is the response function, given
by (\ref{eq:defS}).
% objects:
The appearance of 
$\ME<\alpha|\opOhatless|\beta>$ in this formula simplifies the calculation.
In the case of Gamow-Teller response the operator 
$\opO=\mbox{\boldmath{$\sigma$}}\tau_\pm$ 
is a spherical tensor 
with multipolarity $\lambda = 1$, and angular momentum coupling results in
the fact that reduced matrix elements should be used:
$\RME<\alpha |\opOhatless |\beta>$. The only combinations of $\alpha$ and
$\beta$ are the ones that can be coupled to $\lambda$ 
($|j_\alpha - j_\beta| \le \lambda \le j_\alpha + j_\beta$). 
% index pairing
Now it is natural 
to define a coding $i=\{\alpha, \beta\}$ for all the combinations that can be
coupled to $\lambda$, so the object
$\langle\alpha\|\opOhatless\|\beta\rangle$ can be seen as a vector 
$\rvec{v} = \wveci{v}_i$.

% matrix equation
With this coding, (\ref{eq:DRPA}) is rewritten in the matrix form (\cf\ 
(\ref{eq:RPAinv})
 using $L_{\alpha \beta\gamma\delta}=\rvec{L}=\wveci{L}_{ij}$ 
\etc)
%
	\begin{equation}
		\left(\rvec{L}^{\phantom{f}}(\omega)\right)^{-1}
	= 
		\left(\rvec{L}^f(\omega)\right)^{-1}
	- 
		\rvec{V}(\omega)
	\label{eq:DRPAinv}
	\;.
	\end{equation}
%
The interaction is taken energy-dependent to be able to apply the method to be
described to
the DERPA equation (\ref{eq:DERPA}).
To know the value of the response function at some energy $\omega$ one 
has to solve (\ref{eq:DRPAinv}) together with
%
	\begin{equation}
		S_\opO(\omega) 
	= 
		\rvec{v}^T
		\rvec{L}(\omega) 
		\rvec{v}
	\;,
	\end{equation}
%
where $\rvec{v}^T$ is the transpose of $\rvec{v}$ and $\rvec{v}$ is a vector 
consisting of the matrix elements of $\opO$.
The two methods to solve the RPA equation, 
described in section~\ref{sec:RPAsol}, cannot both be applied.
Only the continuous method will be feasible for D(E)RPA.

%
% continuous method
%
By solving with the continuous method is meant that in (\ref{eq:RPA}) the 
infinitesimal quantity $\eta$ is taken to be finite (\eg\ $0.25$~MeV 
while the range 
of $\omega$ is $0-30$~MeV). This finite `width' $\eta$ is physically observable
when a collection of states is positioned close to each other and therefore 
mix strongly. Isolated states do not appear as $\delta$-peaks (\cf\ 
(\ref{eq:defS})) in experiments, 
but are smeared out also due to finite energy resolution. With a
finite $\eta$ $L$ and $L^{(0)}$ are not singular anymore. 
The wanted $S_\opO$ is found by
solving the following set:
%
	\begin{equation}
		\left\{
		\begin{array}{ccc}
		\rvec{L}^{-1}\rvec{b} & = & \rvec{v}     \\
		\rvec{L}^{-1}         & = & {\rvec{L}^f}^{-1}-\rvec{V}     \\
		S_{GT}                & = & \rvec{v}^T \rvec{b}
		\end{array}
	\right.
	\label{eq:Appset}
	\end{equation}
%
Because the construction of $\rvec{L}^{-1}$ is cheap, due to the diagonal form 
of $\rvec{L}^f$. Solving (\ref{eq:Appset}) is a process of order $n^2$ 
and is hence cheaper than the matrix inversion which is of order $n^3$.

\refrncs
