\chapter{Closed Form of the Moshinsky Brackets\label{app:MTCF}}
The brackets that are used in the Moshinsky transformation 
(\ref{eq:MTapply}) can be written in closed form, as was proven 
for the more general case of two particles with different masses\cite{SG72}.
If the  masses are the same, the expression becomes:
%\footnote%
%
	\begin{eqnarray}
	\label{eq:MTCF}
		\inp< n l N L; \lambda | n_1 l_1 n_2 l_2; \lambda > 
	=
	\end{eqnarray}
%
{ 
\renewcommand{\arraystretch}{1.5}
\tiny
%
	\begin{eqnarray}
	%\label{eq:MTCF}
	%\\
	%\displaystyle
	%	\inp< n l N L; \lambda | n_1 l_1 n_2 l_2; \lambda > 
	%\\
	\begin{array}{c}
	{
	\displaystyle
		2^{-\frac{1}{2} l}
		\left[
		\frac{ (2l_1+1)(2l_2+1)(2L+1)} {(2l+1)}	
		\frac{ (l_1+l_2-\lambda)!}{(l_1+l_2+\lambda+1)!} 
		\frac{ (l+L+\lambda+1)!
		       (l+L-\lambda)!
		       (l-L+\lambda)!
		       (L+\lambda-l)!
                     }{(l_1+\lambda-l_2)! 
                       (l_2+\lambda-l_1)!} 
		\right]^\frac{1}{2}
	}
	\\
	{
	\displaystyle
		\hspace{-7pt}
		(-)^{\lambda+n+n_1+n_2}
		(2l+1)
		\left[
		\frac{ n_1! n_2! n!} {N!}	
		\frac{ 
			\Gamma( n_1+l_1 + \frac{3}{2} )
			\Gamma( n_2+l_2 + \frac{3}{2} )
		     } 
		     {
			\Gamma( n+l + \frac{3}{2} )
			\Gamma( N+L + \frac{3}{2} )
		     }	
		\right]^\frac{1}{2}
		\delta ( 2n+l+2N+L, 2n_1+l_1+2n_2+l_2 )
	}
	\\
	{
	\displaystyle
		\sum_{ \lambda_1 \lambda_2 p_1 p_2 }
		\delta( \lambda_1+\lambda_2, l)
		(-)^{\lambda_2 + 
		     \frac{1}{2}( p_1+p_2+l_1+l_2+\lambda_1+\lambda_2} )
		\;
		2^{-\frac{1}{2}(p_1+p_2)}
	}
	\\
	{
		\frac{
			(l_1+p_1-\lambda_1)!
			(p_1+p_2-L)!
			(l_2+p_2-\lambda_2)!
			(\frac{1}{2}(l_1+p_1+\lambda_1))!
			(\frac{1}{2}(p_1+p_2+L))!
			(\frac{1}{2}(l_2+p_2+\lambda_2))!
		     }
		     {
			(l_1+p_1+\lambda_1+1)!
			(p_1+p_2+L+1)!
			(l_2+p_2+\lambda_2+1)!
			(\frac{1}{2}(l_1+p_1-\lambda_1))!
			(\frac{1}{2}(p_1+p_2-L))!
			(\frac{1}{2}(l_2+p_2-\lambda_2))!
		     }
	}
	\\[5pt]
	{
	\displaystyle
	\hspace{-7pt}
		\frac{
			(2p_1+1)(2p_2+1)
		     }
		     {
			(\frac{1}{2}(l_1+\lambda_1-p_1))!
			(\frac{1}{2}(p_1+\lambda_1-l_1))!
			(\frac{1}{2}(p_1+L-p_2))!
			(\frac{1}{2}(p_2+L-p_1))!
			(\frac{1}{2}(l_2+\lambda_2-p_2))!
			(\frac{1}{2}(p_2+\lambda_2-l_2))!
		     }
	}
	\\[5pt]
	{
	\displaystyle
		\sum_{ x y }
		\frac{
			(-)^{x+y}
			(2p_1-x)!
			(2l_2-y)!
			(p_2+L-p_1+x)!
			(l_1+\lambda-l_2+y)!
		     }
		     {
			x! y!
			(p_1+p_2-L-x)!
			(l_1+l_2-\lambda-y)!
			(p_1-l_2+\lambda-\lambda_1+y-x)!
			(l_2-p_1+L-\lambda_2+x-y)!
		     }
	}
	\\
	{
	\displaystyle
		\sum_{ m_1 m_2 }
		(-2)^{-(m_1+m_2)}
		\frac{ 
			\Gamma( \frac{3}{2} )
			\Gamma( \frac{1}{2}(p_1+p_2+L)+m_1+m_2+\frac{3}{2} )
		     } 
		     {
			\Gamma( p_1+m_1 + \frac{3}{2} )
			\Gamma( p_2+m_2 + \frac{3}{2} )
		     }	
	}
	\\
	{
	\displaystyle
		\frac{ 
			(\frac{1}{2}(p_1+p_2-L)+m_1+m_2)!
		     } 
		     {
			m_1! m_2!
			(n_1-m_1-\frac{1}{2}(\lambda_1+p_1-l_1))!
			(n_2-m_2-\frac{1}{2}(\lambda_2+p_2-l_2))!
			(\frac{1}{2}(p_1+p_2-L)+m_1+m_2-N)!
		     }	
	} 
	\end{array}
	\nonumber
	\end{eqnarray}

%
}
\noindent Summations are restricted by
the triangular conditions imposed by the nine-j symbol
%
\renewcommand{\arraystretch}{1.0}
	\begin{equation}
	\negenJ( p_1, p_2, L, \lambda_1, \lambda_2, l, l_1, l_2, \lambda )
	\mbox{\hspace{10pt}and by\hspace{10pt}}
		\left.
			\begin{array}{rcl}
				p_1+\lambda_1+l_1 
			&=& 
				\mbox{even} 
			\\
				p_2+\lambda_2+l_2 
			&=& 
				\mbox{even} 
			\\
				p_1+p_2+L
			&=& 
				\mbox{even} 
			\end{array}
		\right\}
	\end{equation}
%
\refrncs
