\chapnonum{Acknowledgement}
The author wishes to express his gratitude to all people that have contributed
to this thesis. They will not all be mentioned. Only people who have 
contributed a measurable amount directly
to this thesis will be mentioned below. 
Others, such as the graduate students and post docs of the theory group, 
will not be mentioned by name, although their
collective contribution has been appreciated.

\vspace{5pt}
\noindent%
Klaas Allaart has done an enormous amount of reading, re-reading, 
re-re-reading {\em etc.} in his function as copromotor. 
His persistent point of view
that the thesis should be readable to other people has left its marks on the
text.

\noindent%
Egbert Boeker has kept a tight grip on the time schedule. His efforts resulted
in the finishing of this work in time, within acceptable parameters.

\noindent%
Benno van den Brink provided the author with a complete collection 
files to typeset a thesis in \LaTeXe. 

\noindent%
Wim Dickhoff checked the results of the calculations during the his visits
to the Vrije Universiteit. His presence in the group has been most inspiring.

\noindent%
Aloys Geurts performed the heavy task of proof-reading the document without
any knowledge of quantum mechanics at all. 

\noindent%
Carlotta Giusti has provided additional information on the \eepp\ 
cross-section.

\noindent%
Willem Hesselink and Gerco Onderwater are faced with the task to actually 
{\em measure} the \eepp\ cross-section and have always shown stimulating 
interest in the work on the two-proton spectral function.

\noindent%
Wybo Houkes has written a first version of the package to calculate the 
two-proton spectral function.

\noindent%
Herbert M\"uther has calculated the G-matrix and defect functions that have 
been used throughout this thesis. Also his personal interest in this work has
been inspiring. 

\noindent%
Steve Pieper has provided the correlation functions for the Argonne potential
which have been used in chapter~\ref{chap:PPO} and~\ref{chap:CROSS}.

\noindent%
G\"ustl Rijsdijk has written a computer code to perform the dressed RPA 
calculation. This code had been the basis for the present DRPA library.

\vspace{5pt}
\noindent%
Finally I would like to thank my parents. 
Although they have not contributed to this thesis directly in their function 
as parents, they have satisfied some necessary boundary conditions.

\vspace{20pt}
\noindent%
{\em Wouter Geurts, \\
November 1995}
