\chapnonum{Samenvatting in het Nederlands}
\par\noindent%
\centerline{\large\SectFont Lange- en Korte-Dracht Correlaties in Atoomkernen}
\par\noindent%
\centerline{\large\SectFont een Aanpak met Greense Functies}
\selectlanguage{dutch}
\par\noindent%
Een atoomkern is een complex systeem met een afmeting van enkele femtometers
($1\mbox{ fm}=10^{-15}\mbox{ m}$). Op femtometer schaal is de atoomkern goed
te beschr{\ij}ven als een veel-deeltjes systeem opgebouwd uit z.g. nucleonen 
(protonen en neutronen). Als men dieper zou k{\ij}ken, zou men kunnen
waarnemen dat ook nucleonen een substructuur hebben. Op kleinere schaal, het
domein van de deeltjes-fysica, worden nucleonen beschreven als een 
veel-deeltjes systeem bestaande uit z.g. quarks en gluonen.
Dit dient men te allen t{\ij}de in gedachte te houden, als men de 
beschr{\ij}ving
van atoomkernen in termen van protonen en neutronen wil uitbreiden.
Voor een afdoende beschr{\ij}ving van de eigenschappen van atoomkernen, dient 
men ook te weten wat de onderlinge invloed van de nucleonen is. Op de femtometer
schaal is het zo, dat z.g. sterke wisselwerking of kernkracht overheerst 
over de electromagnetische wisselwerking. Het nare van deze kracht is dat,
in tegenstelling tot de electromagnetische kracht, de precieze vorm
niet bekend is. 
Het beschr{\ij}ven van de kernkracht door middel van  
de juiste potenti\"ele energie (de potentiaal) is dus een kunst, die door 
gespecialiseerde groepen nog steeds wordt beoefend. Het gevolg is, dat er 
verschillende versies van de z.g. nucleon-nucleon ($NN$) potentiaal 
verkr{\ij}gbaar z{\ij}n. Voorbeelden hiervan z{\ij}n  
de Bonn-, N{\ij}megen- en Par{\ij}s-potentiaal.

De vorm van deze realistische potentialen is van een b{\ij}na ontmoedigende 
complexiteit. Het veel-nucleon systeem (als `veel' groter is dan drie) is dan 
ook niet exact oplosbaar. Toch bl{\ij}kt het mogel{\ij}k te z{\ij}n het 
probleem te reduceren.
Als men de wisselwerking van een zeker nucleon met alle andere nucleonen
op een bepaalde manier middelt, kr{\ij}gt men een beschr{\ij}ving van 
nucleonen die
in een gemiddelde potentiaal bewegen. Deze benadering met een gemiddelde 
potentiaal wordt in het Engels Mean Field Approximation (MFA) genoemd.
Deze MFA bl{\ij}kt zo goed te werken dat een verbeterde berekening van de 
kernstructuur de MFA als
uitgangspunt kan nemen.

De quantum-mechanica levert als beschr{\ij}ving van \'e\'en deeltje in een 
(aantrekkende) potentiaal een groot aantal mogel{\ij}ke toestanden op 
waarin dat deeltje
zich kan bevinden. In de atoom-fysica corresponderen deze toestanden  met de
l{\ij}nen in het (absorptie-)spectrum van een atoom. 
Als men een veel-deeltjes systeem wil beschr{\ij}ven, waarvan alle deeltjes
 zich in een potentiaal
bevinden, levert dit in principe een maximaal aantal (basis-) toestanden voor 
de 
kern dat gel{\ij}k is aan
het aantal toestanden per deeltje tot de macht het aantal deeltjes op. Het z.g.
Pauli verbod, dat geen twee deeltjes zich in dezelfde toestand mogen bevinden,
levert tezamen met andere selectie-regels een reductie van het aantal 
toestanden van de kern. Toch bl{\ij}ft het aantal toestanden veel te 
groot om rechtstreeks de leerboekenmethode (eerst de toestanden uitrekenen en 
vervolgens de observabelen) toe te passen. Deze methode bl{\ij}kt toch nog 
bruikbaar als benaderingen worden gemaakt.
Benaderende versies van een rechtstreekse berekening heten schillen-model 
berekeningen. De MFA methode levert als oplossing een beschr{\ij}ving van de 
kern in termen van nucleonen die zich in schillen bevinden, 
vergel{\ij}kbaar met 
de elektronen in een atoom. In de zoektocht in hoeverre de realistische 
potentiaal afw{\ij}kt van de MFA potentiaal, worden slechts enkele van de vele 
schillen in de berekening betrokken.

Een andere aanpak, die het probleem van de enorme hoeveelheid basistoestanden 
omzeilt, is het direct berekenen van 
observabelen. In plaats van het bepalen van  de toestanden van de kern
worden de meetbare grootheden in verband gebracht met z.g. Greense functies.
Het berekenen van deze Greense functies is niet eenvoudiger dan het bepalen
van alle toestanden van de kern, maar de Greense functies kunnen 
met fysisch inzicht worden benaderd. De Greense functies worden ontwikkeld
in termen die allemaal een fysisch proces voorstellen. Dit wordt 
gevisualiseerd
door middel van z.g. Feynman diagrammen. 
Deze aanpak, die ook in gebieden als vaste-stof fysica en theoretische chemie
wordt gebruikt, bl{\ij}kt zeer goed te werken en stelt ons in staat te onderzoeken
welke processen een dominante rol spelen in de dynamica van de nucleonen in
een kern. 
Dit z{\ij}n de redenen dat de Greense functie methode is toegepast in dit 
proefschrift.

In hoofdstuk~\ref{chap:GREEN} is een samenvatting gegeven van het werk
dat voorafging aan het onderzoek waarop dit proefschrift berust. Enig resum\'e
is gewenst omdat de definities en notaties van Greense functies van auteur tot 
auteur verschillen. De verschillende Greense functies die in de andere 
hoofdstukken z{\ij}n gebruikt, z{\ij}n hier de revue gepasseerd. Ook de 
basisvergel{\ij}kingen 
---de Dyson vergel{\ij}king voor de \'e\'en-deeltjes
Greense functie en de Bethe-Salpeter vergel{\ij}king voor de deeltje-gat en de 
twee-deeltjes Greense functie--- met de meest gangbare benaderingen z{\ij}n
getoond.
Tevens is in dit hoofdstuk aangegeven waar het formalisme
uitgebreid kan of moet worden. De andere hoofdstukken van dit proefschrift 
z{\ij}n geschreven als uitbreidingen van hoofdstuk~\ref{chap:GREEN}.

In hoofdstuk~\ref{chap:DERPA} is het formalisme aangewend om  de z.g. 
ladings-ruil respons
van de kern $^{48}$Ca te berekenen. Het gaat hier om het gedrag van
de Calcium kern als een proton wordt vervangen door een neutron 
(of andersom). Dit is een subproces van het b\`eta-verval. 
In eerder werk was al een benadering gevonden, welke een goed resultaat gaf. 
In hoofdstuk~\ref{chap:DERPA} is onderzocht of \'e\'en van de meer 
ingewikkelde processen ook 
een significante invloed heeft. Dit bl{\ij}kt niet het geval te z{\ij}n.

Hoofdstuk~\ref{chap:SPECFAC} is in principe onderzocht in hoeverre het 
schillen-model kan worden gebruikt om de kern $^{16}$O te beschr{\ij}ven.
De zogenaamde \'e\'en-deeltjes spectrale functie werd berekend. 
Het is al een t{\ij}d bekend, dat de MFA een aardige beschr{\ij}ving van een kern 
geeft, 
maar in experimenten waar men \'e\'en nucleon (in dit geval een proton) 
uit de kern schiet, bl{\ij}kt dat dit nucleon in een gecompliceerdere 
toestand
in de kern zat. De beschr{\ij}ving van deze gecompliceerde toestand 
was nog niet
optimaal. Effecten van de korte afstand afstoting van de kernkracht waren nog 
niet voldoende in rekening gebracht in die berekening waarin tevens de 
effecten met langere dracht werden berekend. In dit hoofdstuk z{\ij}n
deze effecten bestudeerd.
Deze effecten van korte afstand afstoting, in het 
Engels aangeduid met short-range
correlations (SRC), z{\ij}n ook onderzocht in de resterende hoofdstukken.

In de hoofdstukken \ref{chap:ppknock} tot en 
met~\ref{chap:CROSS} z{\ij}n berekeningen
gedaan aan waarsch{\ij}nl{\ij}kheden waarmee z.g. drievoudige-co\"{\i}ncidenties
zouden plaatsvinden. In de drievoudige-co\"{\i}ncidentie experimenten 
wordt een elektron op een 
trefplaat geschoten. Dit elektron botst vervolgens op een atoomkern
in de trefplaat en het verstrooide elektron wordt gedetecteerd tezamen met twee 
uitkomende protonen. De filosofie achter deze experimentele opstelling is dat,
daar een elektron in goede benadering \'e\'en proton `raakt', de twee protonen
op de een of andere manier gecorreleerd z{\ij}n. Het idee is dat, door de sterke 
afstoting op korte afstand in de $NN$-potentiaal, in
drievoudige-co\"{\i}ncidentie gebeurtenissen het elektron een tweetal protonen 
vlak na een
botsing treft. De impulsen van de protonen z{\ij}n dan b{\ij} 
benadering even groot en 
tegengesteld.
Het geraakte proton vliegt de kern uit, en het andere heeft ook een te grote
impuls om in de kern te bl{\ij}ven. Berekeningen aan dit soort reacties 
w{\ij}zen uit
dat verschillende potentialen verschillende voorspellingen doen. Doch, het 
verschil is niet zo daverend en het is mogel{\ij}k dat allerlei verhullende 
effecten het verschil teniet doen. Een effect dat niet is meegenomen in de 
huidige berekeningen is het feit, dat de protonen op hun weg naar buiten de 
kern nog voelen. Ook z{\ij}n er andere reactie-mechanismen mogelijk, zoals
botsing met een pi-meson of aanslag van de interne vr{\ij}heidsgraden van het
nucleon.
De hier berekende kernstructuur zal met de reactie-mechanismen moeten worden 
ge\"{\i}ntegreerd in een totale berekening.
