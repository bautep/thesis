\section{Computational Procedure\label{sec:SPcompset}}
\subsection{Model Space and Interaction}
The shell model space in which the effective Hamiltonian 
(\ref{eq:Hamiltonian}) acts is chosen to consist of the first four harmonic 
oscillator shells. These orbits
are excluded by the Pauli operator in the Bethe-Goldstone equation 
(\ref{eq:BGEG}),
which was used to construct the G-matrix interaction employing the Bonn-C 
potential\cite{Ma89}. 
The corresponding HF single-particle energies 
$\varepsilon_\alpha(\omega)$ are shown in fig.~\ref{fig:Ehf}. 
For the 
orbits around the Fermi level ($p$-shell and $sd$-shell) they are slightly 
(with $1-2$~MeV)
adjusted, such that the total calculation (HF $+$ 2p1hTDA) reproduces the 
experimental energies of the quasiparticle states, \ie\ the states with the 
large spectroscopic factors.
This adjustment has  been done by shifting the 
$\varepsilon_\alpha(\omega)$ curves of fig.~\ref{fig:Ehf}
in vertical direction. Thereby the normalization effect (\ref{eq:sfNorm}) of
the derivative was left unchanged. The single-particle states and the obtained
quasiparticle energies are listed in table~\ref{tab:sps}. 
%%%%%%%%%%%%%
\begin{table}
\centerline{
\begin{tabular}{|cr@{.}lr@{.}l|cr@{.}lr@{.}l|}
 \hline
 \multicolumn{1}{|c}{ }     & \multicolumn{4}{c}{Energy} &
 \multicolumn{1}{|c}{ }     & \multicolumn{4}{c|}{Energy} \\
 \multicolumn{1}{|c}{Shell} & \multicolumn{2}{c}{proton} 
                            & \multicolumn{2}{c}{neutron} &
 \multicolumn{1}{|c}{Shell} & \multicolumn{2}{c}{proton} 
                            & \multicolumn{2}{c|}{neutron} \\
 \hline
 $1s\frac{1}{2}$ & $-35$&$0$&$-40$&$0$ & $2s\frac{1}{2}$ & $-0$&$1$&$-3$&$3$\\
 $1p\frac{3}{2}$ & $-18$&$5$&$-21$&$8$ & $1f\frac{7}{2}$ & $17$&$4$&$14$&$0$\\
 $1p\frac{1}{2}$ & $-12$&$1$&$-15$&$7$ & $1f\frac{5}{2}$ & $23$&$5$&$20$&$2$\\
 $1d\frac{5}{2}$ & $ -0$&$6$&$ -4$&$1$ & $2p\frac{3}{2}$ & $16$&$0$&$12$&$4$\\
 $1d\frac{3}{2}$ & $  4$&$4$&$  0$&$9$ & $2p\frac{1}{2}$ & $17$&$7$&$14$&$3$\\
 \hline
\end{tabular}
}
\caption[]{The single-particle states that belong to the model space $\calM$,
which was used in calculations on $^{16}$O.
\label{tab:sps}}
\end{table}
%%%%%%%%%%%%%

\subsection{Calculation of the 2p1hTDA Self-Energy}
In principle the calculation of the eigenvalues and eigenvectors of the 
Hamiltonian 
within the 2p1h and 2h1p spaces, leading to $\bar{R}$ (\ref{eq:Risbb}) 
in TDA, should be done with the energy-dependent G-matrix elements and 
single-particle energies. This is a cumbersome procedure and in fact 
unnecessary for practical purposes. A simpler procedure is justified by the 
following two observations.  First, the $\omega$-dependence of the G-matrix 
and the HF-energies is weak and smooth as compared to the pole structure of 
$\bar{R}$ (\ref{eq:Risbb}). So a large amount of computational effort may be 
saved by calculating the poles and eigenvectors in (\ref{eq:Risbb}) with a 
fixed `starting energy' for the G-matrix, for which a value of 
$-40$~MeV was taken
as a suitable average for a 2h1p state. For the 2p1h
states one would prefer  a larger starting energy. These states are less 
important for the 
description of the experimentally measured hole spectral function.
The second observation is that
the calculation of the self-energy and the procedure of solving the Dyson 
equation become effectively decoupled 
when the condition is imposed that the outcome of the quasi-particle energies 
for the shells around the Fermi level must correspond to the experimental ones,
So the method adopted here is to
calculate the eigenvalues $\omega^\nu$ and eigenvectors $b^\nu_{lmn}$ with the 
quasi-particle energies of table~\ref{tab:sps} and G($\omega=-40$~MeV).
Next, the Dyson equation (\ref{eqbis:DysonE}) is solved with the self-energies 
(\ref{eq:SigHFe}) and (\ref{eq:SigTDA}). 
If the energies of the main quasi-particle peaks around the Fermi level differ 
appreciably from the experimental values, the corresponding single-particle 
energy curve $\varepsilon^{HF}_\alpha(\omega)$ 
is shifted and the whole procedure is 
repeated. As a result the quasi-particle values of table~\ref{tab:sps} were 
obtained. Only a few iterations are needed because the adjustments are small.  
Moreover, high precision at this point is not meaningful in view of the 
various approximations and slight inconsistencies that are still inherent in 
the method and which are shortly addressed in the next section.

\subsection{Lack of Self-Consistency}
Besides the use of a fixed $\omega$-value for the G-matrix interaction in the 
calculation of the 2p1h and 2h1p amplitudes, there is a more important problem 
if one wishes to solve the Dyson equation with a self-consistent self-energy. 
This arises as soon as the 2p1h-propagator in (\ref{eq:SigTDA}) is no longer 
restricted to its Faddeev (TDA) form, but consists of diagrams that involve
`dressed' nucleons, \ie\ nucleons fully interacting with all other nucleons.
These dressed nucleons are then described with a fragmented one-body propagator
and no longer with the single-pole propagator of (\ref{eqbis:gHF}).
Now the self-energy (\ref{eq:SigTDA}) depends on the solution of the 
Dyson equation in which it appears. The use of an iterative scheme, where the
propagator of the $(n-1)$-th step is used as input for the calculation of the
self-energy in the $n$-th step is not possible, because  in each step, 
the number of poles increases and thereby the complexity of 
the expressions for the self-energy.
The self-consistency can only be studied for very simple approximations of the self-energy, \eg\ independent propagation of the particles and holes in 
(\ref{eq:gR}), or with a simple form for the interaction.
This has been performed by 
Van Neck\cite{NWR91}, 
who represented the propagators in a limited set of energy bins.
A different approach to reach a certain self-consistency is the 
BAsis GEnerated by the Lanczos (BAGEL)\cite{MS93} method.
In this method the Dyson equation, with a self-energy (\ref{eq:SigTDA}) but 
with 
independent propagation of the particles and  holes, is written as a matrix 
within the space composed of
the one-hole state and the  2h1p states. The most important 
eigenvectors are then 
filtered out by the Lanczos method.
A self-consistent calculation of the one-body propagator 
for the Tin isotopes using a pairing force 
and with a 2p1hTDA 
self-energy 
is given in ref.~\cite{Yu94}.


As a consequence of the lack of self-consistency elementary sum rules may be 
violated. For a given orbit the sum of all removal and addition spectroscopic 
factors, \ie\ of all residues of the corresponding Green's functions, must be 
unity. In the present case, however, this sum is already smaller in HF 
approximation. As a consequence of 
the energy-dependence of the G-matrix the strength at 
very high energies due to short-range correlations is hidden. So to assess the 
violation of the sum rule due to the lack of self-consistency the result of 
the 
calculation without any energy-dependence in the HF-part should be considered.
In this case the violation appeared to be only of the order of $10^{-8}$.
