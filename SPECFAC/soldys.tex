\section{Results and Discussion\label{sec:SPsoldys}}
The  gross features of the spectral functions obtained by the method of 
section~\ref{sec:SPcompset} are displayed in  fig.~\ref{fig:sfO}.
%%%%%%%%%%%%%%
\begin{figure}
\centerline{%
\hspace{-0.8cm}%
\epsfig{ figure=figures/sfO.ps, width=11.5cm, height=10.5cm }
}
\caption[]{Residues of single-particle Green's functions (\ref{eq:g1s}).
Bottom to top, the quantum numbers ($\alpha$) of the states are
 $1s\frac{1}{2}$,
 $1p\frac{3}{2}$,
 $1p\frac{1}{2}$,
 $1d\frac{5}{2}$,
 $1d\frac{3}{2}$,
 $2s\frac{1}{2}$,
 $1f\frac{7}{2}$,
 $1f\frac{5}{2}$,
 $2p\frac{3}{2}$ and
 $2p\frac{1}{2}$. Below $-10$~MeV the hole strength, \ie\ particle removal 
spectroscopic factors, is found while above the Fermi level the height of the 
spikes indicate the particle addition spectroscopic factors.
Note that the vertical scales are logarithmic.
\label{fig:sfO}}
\end{figure}
%%%%%%%%%%%%%%
The figure exhibits the well-known features of a quasiparticle state clearly 
sticking out for orbits around the Fermi energy and a much more fragmented 
distribution for those orbits further away from the Fermi level. A more 
detailed comparison with the spectroscopic strength deduced from \eep\ 
data\cite{Leu94} is made in fig.~\ref{fig:Leus}.
%%%%%%%%%%%%%%
\begin{figure}
\centerline{%
\hspace{-0.8cm}%
\epsfig{figure=figures/sfObin.ps, width=11cm}}
\caption[]{
The calculated hole spectral function (weighted with a factor 
\mbox{$2j_\alpha$$+$$1$}) for
different $l$-values (left) is compared with the measured one-hole spectral 
function of $^{16}$O taken from ref.~\cite{Leu94} (right). 
\label{fig:Leus}}
\end{figure}
%%%%%%%%%%%%%%
This figure clearly shows, that the calculated spectroscopic factors for the 
lowest $\half^-$ and $\threehalf^-$ states ($l=1$) are still too large by about 
$15-20$\% of the independent-particle shell model values. 
This discrepancy would 
have been seven percent larger if the energy-dependence of the G-matrix, 
simulating the effect of short-range correlations, had been neglected. 
This comparison is shown in table~\ref{tab:sfO}.
%%%%%%%%%%%%%%
\begin{table}
\centerline{
\begin{tabular}{|c|lll|lll|}
\hline
\multicolumn{1}{|c|}{}         & \multicolumn{3}{c|}{Energy dependent} 
         & \multicolumn{3}{c|}{Energy independent} \\
\cline{2-4} \cline{5-7}
 \multicolumn{1}{|c|}{Shell}
         & \multicolumn{1}{c}{hole}
         & \multicolumn{1}{c}{particle}
         & \multicolumn{1}{c|}{main peak}
         & \multicolumn{1}{c}{hole}
         & \multicolumn{1}{c}{particle}
         & \multicolumn{1}{c|}{main peak} \\
\hline
 $1d\frac{3}{2}$ & 0.035  &0.914  &0.82  &0.037 &0.963   &0.87   \\
 $1d\frac{5}{2}$ & 0.034  &0.913  &0.86  &0.037 &0.963   &0.91   \\
 $1p\frac{1}{2}$ & 0.837  &0.094  &0.77  &0.903 &0.097   &0.83   \\
 $1p\frac{3}{2}$ & 0.879  &0.052  &0.76  &0.945 &0.055   &0.82   \\
 $1s\frac{1}{2}$ & 0.880  &0.043  &---   &0.954 &0.046   &---    \\
\hline
\end{tabular}}
\caption[]{Comparison of the solution of the Dyson equation with and without
influence of the energy-dependence of the G-matrix.
\label{tab:sfO}}
\end{table}
%%%%%%%%%%%%%%

Other authors\cite{NDG95}, applying a local density approximation, find a 
reduction of the spectroscopic factor due to short-range correlations of more 
than $20$\% and therefore, putting surface vibrations on top of that, 
obtain too small spectroscopic factors. 
The results of ref.~\cite{NDG95} may be an 
indication that short-range correlations are not sufficiently dealt with by 
the present approach,
although their effect calculated in the present work agrees with more direct 
calculations\cite{MPD95,MD94}. However, the calculations in 
ref.~\cite{RBP94} indicate that the quasi-hole peak is reduced by another 
$10$\% when the center-of-mass motion is treated properly.
Fig.~\ref{fig:Leus} shows that the fragmentation of the $l=0$ 
strength as well as the side peaks for $l=1$ are underestimated by the 
calculations. This has clearly nothing to do with SRC, but rather with the 
still inadequate treatment of long-range correlations. It is not easy, however, 
to find a better, numerically still feasible, approximation for the 
irreducible 2p1h propagator (\ref{eq:gR}) and thereby for the self-energy. 
The search for such an approximation should focus on complex structure at 
low energies\cite{RAD92,RGAD95}, because the discrepancy between calculations
and data is less for the integrated strength over the whole energy range up to 
$40$~MeV. The data\cite{Leu94} gave $4.30$ for $l=1$ and 
$0.254$ for $l=2$ as compared to $4.78$ and $0.261$ for the 
calculations.
The calculated spectral function for $l=2$ in fig.~\ref{fig:Leus} agrees 
rather nicely with the data, as its non-zero value is entirely due to 
correlations. Also for these orbits a larger role of collectivity at low 
energy would give a slight further improvement, \viz\ in this case a stronger 
concentration in the lowest (collective) state would be achieved.

