\section{Introduction}
In several \eep\ experiments%
\cite{HBJ88,Qu88a,Kr90,Ste91} a substantial 
fragmentation of the one-nucleon knock-out strength has been observed. 
In various calculations of the one-nucleon spectral 
function at low energies\cite{BRM91,RGBA93,NWR91,MS93} 
this fragmentation has been 
shown to be 
due to long-range correlations. It was also noticed, however, that an 
additional mechanism must be acting which reduces the spectral strength in the 
low-energy region, especially the (main) quasiparticle peak. 
The effect of short-range 
correlations represents a prime candidate for the explanation of 
this discrepancy between 
the calculations and the data. 
Calculations in nuclear matter\cite{RPD89} show that this effect is of order
$10-15$\%.

In other theoretical studies the spectral function calculated for nuclear 
matter was transformed to the finite nucleus by a 
local density approximation\cite{NDG95}. 
In these calculations spectroscopic 
factors for the states around the Fermi level 
came out even considerably lower than 
deduced from the data. This is possibly due to double counting 
when the effect of surface vibrations deduced from phenomenological optical 
potentials is added on top of that deduced from nuclear matter.

The quasi-hole wave function of $^{16}$O was also calculated with variational 
methods\cite{RBP94}. These results were compatible with a 
reduction of the spectroscopic factor depending on whether center-of-mass 
motion was taken care of ($20$\%) or not ($10$\%). 
In Brueckner-Hartree-Fock 
calculations\cite{MD94} neglecting center-of-mass motion a value of $0.91$ 
for the hole spectroscopic factor was found.

The reason why long- and short-range correlations were not dealt with 
simultaneously in a 
calculation for a finite nucleus so far 
is that an excessively large (shell) model 
space would be required to include the scattering by the strongly repulsive 
cores of a realistic $NN$-interaction. For this reason one normally deals with 
a limited model space, of say four or five major shells, and a Brueckner 
G-matrix as an effective interaction. By construction the G-matrix, which is 
the  solution of a Bethe-Goldstone equation, is energy dependent. This energy 
dependence is rather weak, however, as compared to that of the dispersion 
effects which at low energies contribute pole terms to the self-energy. 
Therefore, in low energy nuclear structure calculations the G-matrix is 
usually considered as a static, \ie\ energy independent interaction. In the 
present chapter,
however, it is our aim to take its energy-dependence into account 
and study its effect on the spectral function at low energies. It may be 
expected that, by doing so, one accounts to a good approximation for the 
effects of the short-range correlations, which were treated in the 
construction of the G-matrix. In this sense our results do incorporate 
effects of both short- and long-range correlations consistently.

The computational procedure and input are described in 
section~\ref{sec:SPcompset}, results for the one-body spectral function for
$^{16}$O are presented and compared with the available data
in section~\ref{sec:SPsoldys}.
