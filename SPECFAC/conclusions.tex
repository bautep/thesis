\section{Summary}
In this chapter the effect of short-range correlations has been included in the 
calculation of the one-proton removal amplitudes by taking into 
account the energy-dependence of the G-matrix. Due to this 
energy-dependence an extra depletion by somewhat less than $10$\%
has been found. This feature is in agreement with a direct calculation 
for $^{16}$O\cite{MPD95}.
The fragmentation by long-range correlations has been described by the 
coupling to 
the 2h1p and 2p1h propagator in Tamm-Dancoff approximation. The obtained 
fragmentation is too small for the $l=1$ strength which signals that not yet 
all the relevant low-energy dynamics is adequately incorporated.
This has to be expected in view of the complicated excitation spectrum of the 
initial nucleus ($^{16}$O). 
The  total $l=2$ hole strength below $40$~MeV missing energy compares well 
with the data. The fragmentation of the deep-lying $l=0$ hole state probably 
requires  continuum description of the final nucleus, as its missing energy 
spectrum peaks well above the two-nucleon emission threshold. 
If the calculated spectroscopic factors for the $\frac{1}{2}^-$ and 
$\frac{3}{2}^-$ quasi-hole states are reduced by another $10$\%, to account 
for center-of-mass motion\cite{RBP94}, then these quantities agree with the 
data to within $10$\%
The calculations might be improved using a larger model space for the
long-range correlations. Also a `Faddeev' calculation in full space would be
interesting. Extensions of the Faddeev approximation, to incorporate RPA like
(backward going) diagrams would be interesting too, but inclusion of such 
diagrams is far from straightforward\cite{Rij93}.
