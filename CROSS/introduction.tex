\section{Introduction}
A calculation of the cross-section for two-proton emission should contain
the interaction of the electron with the nucleus, the internal structure of the
nucleus and the final state of the ($A-2$)-nucleus with two out-coming protons.
As observed in other electron scattering experiments, the interaction of the 
electron with the nucleus can be written in good approximation as the exchange
of one virtual photon. In this case the electron part of the cross-section is
known, and all the unknown properties of the nucleus are in the so-called 
hadronic part of the cross-section. This hadronic part arises from the 
interaction Hamiltonian that couples the (virtual-)photon
field to a nuclear current. The factorization widely used in the interpretation
 of
\eep\ results stems from the fact that a one-body (electro-magnetic) current 
operator can be used to describe the hadronic part of the cross-section.

In the \eepp\ calculation both one-body and two-body currents should in 
principle be 
incorporated in the hadronic part. Several groups \cite{Ry94,GP91,GP92,GP94}
adopt some form of two-body current. If the two-protons were knocked out by just
a one-body contribution, the effect would exclusively be due to short-range 
correlations. 
Therefore, if one is interested in studying these correlations, it must also  
be investigated what the effects of two-body
currents (\eg\ Meson exchange currents (MEC) or isobar currents (IC)) are.

The effect of the FSI are not treated in this chapter, because the focus is
on the effects of the short-range parts of  different realistic $NN$-potentials.
Without FSI the out-coming protons are in plane wave states. A  
study\cite{GP91} of 
FSI effects shows that different optical potential models, used to 
describe 
these effects, yield a reduction of the total cross-section with an average 
factor of $2.5$ with a spread of about $0.5$.

In this chapter the spectral function calculated in chapters \ref{chap:ppknock} 
and~\ref{chap:PPO} is used as input for the calculation of the longitudinal 
cross-section in plane wave approximation. For that purpose a review of the
ingredients of the cross-section is given in section~\ref{sec:CROSStheory}.
In section~\ref{sec:CROSSresults} two kinematical settings are chosen to display
the results for the cross-section. Conclusions are drawn in 
section~\ref{sec:CROSSconclusions}.
