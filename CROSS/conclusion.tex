\section{Conclusions\label{sec:CROSSconclusions}}
The longitudinal part of the \eepp\ cross-section is proportional to the 
two-proton spectral function, if some approximations are adopted. These are 
the spectator approximation, \ie\ interaction of the probe with the 
knocked-out nucleons only\cite{GP91}, restriction of current operator to the 
one-body (charge) terms and the omission of the final state interaction of the 
ejected particles with each other or with the residual nucleus (plane-wave 
approximation).

With the spectral functions of chapter~\ref{chap:PPO} this cross-section is 
calculated for several kinematical settings of the experimental setup. 
Perhaps the most relevant result is the roughly factor two difference between 
the cross-sections with the Bonn and Reid potentials for these angles where 
large relative momenta give the dominant contribution. This factor two is well 
beyond the uncertainty due to distortion of the outgoing waves\cite{GP91}. The 
differences with the older KK and OMY potentials as well as with the 
variational wave functions of ref.~\cite{PWP92} are much larger and therefore 
it should be relatively easy to discriminate experimentally between those and 
the Bonn and Reid potentials.

Of course it should be kept in mind that also other mechanisms may contribute 
to the cross-section, like charge-exchange in the final state, meson exchange 
and isobar currents. These may especially contribute in the transverse part of 
the cross-section, as seen in ($\gamma,pp$) experiments \cite{Ry94}. 
So a longitudinal-transverse separation will be very helpful.

There is some controversy about the importance of the $\Delta_{33}$ isobar on 
the \eepp\ cross-section. According to calculations of the Ghent 
group\cite{Ry94} their contribution is large whereas the Pavia 
group\cite{GP91,GP92,GP94} finds that these are small. An argument in favor of 
the latter result, is that the knock-out of a $^1S_0$ pp-pair via an 
intermediate $^1D_2$ $\Delta^+p$ state is expected to be 
suppressed\cite{BlokPrive}.

Apart from all these uncertainties, which must be investigated further, 
the present results suggest that coplanar kinematics and angles 
$\gamma_{pq} \approx 75^\circ - 80^\circ$ 
are suitable to study the $2^+$ final states. Also for the transition to the 
$0^+$ ground state of $^{14}$C this kinematics in the range
$\gamma_{pq} \approx 60^\circ - 80^\circ$ 
is of interest to discriminate between the modern potentials, Bonn and Reid, 
and the other ones. For  a distinction between the former two a restriction to 
zero center-of-mass momentum and 
$\gamma_{pq} \approx 75^\circ$ 
 may be suitable, although the cross-sections will be small.
