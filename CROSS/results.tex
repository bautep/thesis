%%%%%%%%%%%%%%
\begin{figure}[t]
\centerline{\epsfig{figure=figures/CROSS/GP0.ps, width=10cm}}
\caption[]{
Longitudinal cross-section in plane wave approximation for the transition to 
the ground state of $^{14}$C. The virtual photon
momentum $\rvec{q}$ is directed along the $z$-axis and has a magnitude of
$2$~fm$^{-1}$, the proton momenta are detected at angle $\gamma_{pq}$ and 
both have a magnitude of $1.8$~fm$^{-1}$.  
The beam energy $p\phpr_0$ is $700$~MeV, 
the transferred energy $\omega$ is $150$~MeV and the scattering angle of the 
electron is $34.8^\circ$. Therefore the virtual photon polarization is
$\varepsilon=0.81$. The cross-section is calculated by multiplying the 
spectral function (\ref{eq:Shat}) with 
$K2\varepsilon_L G^2\approx (0.0049)\ast(1.38)\ast(0.48) = 0.0033$ (fm$^2$).
\label{fig:GP0}
}
\end{figure}
%%%%%%%%%%%%%%
\section{Results\label{sec:CROSSresults}}
The expression for the longitudinal cross-section that is proportional to 
(\ref{eq:spect}), depends on three three-momenta. Before this function can 
be visualized, restrictions have to be found. 
In the coplanar set-up also used by ref.~\cite{GP91}, 
the virtual photon momentum $\rvec{q}$ is 
by definition along the $z$-axis, while the two detected protons are in the
$x$--$z$-plane at equal angles with respect to $\rvec{q}$. By varying this 
angle ($\gamma_{pq}$), one is able to vary the relative and center-of-mass
momentum of the pair simultaneously.

To check a consistency with earlier calculations of the cross-section by
ref.~\cite{GP92}, the longitudinal cross-section is calculated in the 
coplanar setting.
Agreement with fig.~1 of ref.~\cite{GP92} for the OMY and KK potential is found.
The results for the transition to the ground state of $^{14}$C are displayed in
fig.~\ref{fig:GP0}. One notes that those for the realistic Bonn and Reid 
potentials do not differ very much, in contrast to the older KK and OMY 
potentials. The correlation functions obtained by Pieper \etal\ in a 
variational procedure with the Argonne $v_{14}$ potential\cite{PWP92} appear 
to yield a small cross-section at large angles, \ie\ at high relative momenta 
of the knocked-out protons. This could already be expected from a comparison of 
the defect functions in fig.~\ref{fig:defectPiep}.
%
The cross-section for the transition to the first $2^+$ is displayed in 
fig.~\ref{fig:GP2}. Here the difference between the the Reid and Bonn 
potential is more than a factor two for the largest angles
$\gamma_{pq} \approx 75^\circ - 80^\circ$. The curves of the cross-section
for the transition to the second $2^+$ are close to the ones of 
fig.~\ref{fig:GP2}. This opens the possibility to investigate the transition 
to the $2^+$, even if one is not able to separate the two $2^+$ states.
%%%%%%%%%%%%%%
\begin{figure}
\centerline{\epsfig{figure=figures/CROSS/GP2.ps, width=10cm}}
\caption[]{
Cross-section for the transition to the first $2^+$ excited state in $^{14}$C.
See caption of fig.~\ref{fig:GP0}.
\label{fig:GP2}
}
\end{figure}
%%%%%%%%%%%%%%
In 
the coplanar setup, the angle is related to the center-of-mass momentum because
the length of the momenta of the out-coming protons are fixed. 
The structure in the curves in figs.~\ref{fig:GP0} and~\ref{fig:GP2} is 
partly due to the dependence on the center-of-mass momentum. The curves can not
directly be compared with \eg\ defect functions.
If one fixes the center-of-mass
momentum to zero, then the angle $\gamma_{pq}$ will only probe the relative
momentum and one obtains functions as displayed in figs.~\ref{fig:cmvar0} 
and~\ref{fig:cmvar2}. 
%%%%%%%%%%%%%%
\begin{figure}
\centerline{\epsfig{figure=figures/CROSS/cmvar0.ps, width=10cm}}
\caption[]{
\label{fig:cmvar0}
Cross-section for the transition to the ground state in the coplanar, 
$P_{CM}=0$ setup. In this case the angle $\gamma_{pq}$ probes the relative 
momentum. The virtual photon momentum $\rvec{q}$ is set to $1.6$~fm$^{-1}$. 
}
\end{figure}
%%%%%%%%%%%%%%
%%%%%%%%%%%%%%
\begin{figure}
\centerline{\epsfig{figure=figures/CROSS/cmvar2.ps, width=10cm}}
\caption[]{
Cross-section for the transition to the first $2^+$ excited state in $^{14}$C.
See caption of fig.~\ref{fig:cmvar0}.
\label{fig:cmvar2}
}
\end{figure}
%%%%%%%%%%%%%%
The $P_{CM}=0$ setting seems to offer a better opportunity to discern the 
different realistic $NN$-potentials if only the transition to the ground state
can be probed. The $2^+$ state can best be analyzed with the coplanar setup.
