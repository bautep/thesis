\section{Theory\label{sec:CROSStheory}}
The theoretical framework of the triple-coincidence cross-sections
for the \eepp\ reaction is given in ref.~\cite{GP91}. All the conventions
used in this chapter are the same as in that work. In summary one has to
calculate the nine-fold differential cross-section depending on the measured
four-momenta $\rvec{p}'_0$, $\rvec{p}'_1$ and $\rvec{p}'_2$ of respectively the
scattered electron and the two out-coming protons. Initial momenta will be 
denoted without a prime. In the one-photon exchange approximation the 
nine-fold differential cross-section is given by
%
	\begin{equation}
		\frac{
			{\rm d}^9 \sigma
		}{
			{\rm d}\rvec{p}'_0
			{\rm d}\rvec{p}'_1
			{\rm d}\rvec{p}'_2
		}
	=
		\alpha^2
		\frac{1}{
			q_\kappa^2 p\phpr_0 p'_0 (\varepsilon - 1)
		}
		L_{\mu\nu} W^{\mu\nu}
	\label{eq:crosss}
	\end{equation}
%
where $\alpha=\frac{1}{137}$ is the fine structure constant, $q_\mu$ is the 
four-momentum transfer $q_\mu=( \omega, \rvec{q} )$ with 
$\omega=p\phpr_0-p'_0$ as the
transferred energy ($p\phpr_0$ and $p'_0$ are the electron energies of 
respectively
the beam and scattered electron) and $\rvec{q}=\rvec{p}\phpr_0-\rvec{p}'_0$ the 
transferred three-momentum. The quantity
%
	\begin{equation}
		\varepsilon
	=
		\left(
		 1 - \frac{2\rvec{q}^2}{q_\nu^2} \tan^2(\half\theta)
		\right)^{-1}
	\end{equation}
%
is the transverse polarization of the virtual photon that is  transferred by 
the electron.
The angle $\theta$ is the angle between $\rvec{p}\phpr_0$ and $\rvec{p}'_0$ 
(in the
laboratory system).

The only unknown quantity in the expression for the cross-section is the 
hadronic
tensor $W^{\mu\nu}$, since the leptonic tensor $L_{\mu\nu}$ is known.
The hadronic tensor is defined as
%
	\begin{equation}
		W^{\mu\nu}
	=
		\ME< \Psi_f | \hat{J}^\mu | \Psi_i >
		\ME< \Psi_f | \hat{J}^\nu | \Psi_i >^\ast
		\delta( E_i - E_f )
	\label{eq:hadron}
	\;,
	\end{equation}
%
where $\hat{J}^\mu$ is the current operator that appears in the interaction 
Hamiltonian term $A_\mu\hat{J}^\mu$, where $A_\mu$ is the photon field. In the 
case that only one-proton is knocked out it is likely that only one-body 
currents will contribute. Calculations of the Ghent group\cite{Ry94}, 
however, also find
a reasonable contribution of two-body current contributions in the \eep\ 
cross-section.In the case of \eepp\ two-body currents (\eg\ meson exchange 
currents (MEC)) may contribute as well.

Usually the cross-section is written in terms of  structure functions as
an eight-fold differential cross-section (one energy variable is eliminated 
by energy conservation). 
The longitudinal part is then
%
	\begin{equation}
		\frac{ {\rm d}^8 \sigma }
		{
			{\rm d} p'_0
			{\rm d} \Omega'_0
			{\rm d} E'_1
			{\rm d} \Omega'_1
			{\rm d} \Omega'_2
		}
	=
		2K\varepsilon_L W^{00}
	\end{equation}
%
with
%
	\begin{equation}
		K
	=
		\alpha^2
		\frac{ \Omega_f }
		{ q_\mu^2 p\phpr_0 p'_0 (\varepsilon -1) }
		\frac{1}{f_R}
	\end{equation}
%
a kinematical factor 
in which $\Omega_f={p'_0}^2 p'_1 E'_1 p'_2 E'_2$ is a phase space factor,
$\alpha$ is the fine structure constant  and
${f_R}^{-1}$ is a recoil factor.
The factor $\varepsilon_L$ is defined as
%
	\begin{equation}
		\varepsilon_L
	=
		-\frac{ q_\mu^2 }
		{ \rvec{q}^2 }
		\varepsilon
	\end{equation}
%
and is called the longitudinal polarization.

\subsection{The Spectator Model}
The matrix elements in the expression for the hadronic tensor (\ref{eq:hadron})
involve many-body  wave functions including the outgoing particles. Because the
focus is on the situation where one has two 
outgoing particles, one could project the final state on the 
two-particle emission channel.
The approximation that will be made here is that the current operator will only
act on the nucleons that leave the nucleus, the residual nucleus acts as
a spectator.
If the residual nucleus
is left in the excited state $\ket{\Psi^{A-2}_n}$, projection on the 
two-particle emission channel gives in the spectator approximation:
%
	\begin{eqnarray}
	\lefteqn{
		J_n^\mu(\rvec{q})
	=} &&
	\\
	&&
		\int
		{\rm d} \rvec{r}
		{\rm d} \rvec{r}_1
		{\rm d} \rvec{r}_2
		\inp< \Psi_f | \Psi^{A-2}_n \rvec{r}_1 \rvec{r}_2 >
		\ME< \Psi^{A-2}_n \rvec{r}_1 \rvec{r}_2 | 
                   \hat{J}^\mu( \rvec{r}, \rvec{r}_1,\rvec{r}_2)
		| \Psi_i >
		e^{i \rvec{q}\cdot\rvec{r}}
	\;.
	\nonumber
	\end{eqnarray}
%
Assuming that the protons leave the nucleus in opposite direction one may
argue that their mutual interaction in the final state is unimportant 
and describe the out-coming protons by the 
eigenfunctions $\zeta$ of an optical potential.
The overlap with the final state can then be rewritten as
%
	\begin{equation}
		\inp< \Psi_f | \Psi^{A-2}_n \rvec{r}_1 \rvec{r}_2 >
	=	
		\zeta_{\rvec{p}'_1}^\ast (\rvec{r}_1) 
		\zeta_{\rvec{p}'_2}^\ast (\rvec{r}_2) 
	\label{eq:finalst}
	\;.
	\end{equation}
%
The overlap matrix elements of the initial states can be written in
second quantized form (assuming that the initial nucleus is in the ground
state) as
%
	\begin{equation}
		\ME< \Psi^{A-2}_n \rvec{r}_1 \rvec{r}_2 | 
                   \hat{J}^\mu( \rvec{r}, \rvec{r}_1,\rvec{r}_2)
		| \Psi_i >
	=
               	J^\mu( \rvec{r}, \rvec{r}_1,\rvec{r}_2)
		\ME< \Psi^{A-2}_n | 
			\Oa{\rvec{r}_1}\Oa{\rvec{r}_2} 
		| \Psi^A_0 >
	\end{equation}
%
where $J^\mu( \rvec{r}, \rvec{r}_1,\rvec{r}_2)$ is in principle the sum over 
one- and two-body contributions.
\subsection{Longitudinal Part in Plane Wave Approximation}
The longitudinal part of the cross-section solely depends on the entry $W^{00}$ 
of the hadronic tensor. In the approximation that the protons are only knocked
out by a one-body current, one has only to take into account the zero-th
element of the current operator
%
	\begin{equation}
		J^0(\rvec{r}, \rvec{r}_1,\rvec{r}_2)
	=
		G^p_E( q_\mu^2 ) ( \delta( \rvec{r}-\rvec{r}_1)
		                  +\delta( \rvec{r}-\rvec{r}_2))
	\end{equation}
%
where $G^p_E( q_\mu^2 )$ is the Sachs proton charge form factor. 
The numerical form used here is the 
parameterization given in ref.~\cite{SSBW80}.
In plane wave approximation the outgoing particles are supposed to be in plane
wave states (instead of states distorted by \eg\ an optical potential), so
the final state overlap function (\ref{eq:finalst}) becomes
%
	\begin{equation}
		\inp< \Psi_f | \Psi^{A-2}_n \rvec{r}_1 \rvec{r}_2 >
	=	
		e^{i\rvec{p}'_1\cdot\rvec{r}\phpr_1} 
		e^{i\rvec{p}'_2\cdot\rvec{r}\phpr_2} 
	\label{eq:finalstpw}
	\end{equation}
%
and the total hadronic tensor for a transition to the $n$-th excited state of
the residual system becomes
%
	\begin{eqnarray}
		W^{00}(\rvec{q})
	&=&
		\left(G^p_E(q_\mu^2) \right)^2
		\left|
		\ME< \Psi^{A-2}_n | 
			\Oa{\rvec{p}'_1-\rvec{q}}\Oa{\rvec{p'}_2} 
		| \Psi^A_0 >
		+
		\ME< \Psi^{A-2}_n | 
			\Oa{\rvec{p}'_1}\Oa{\rvec{p'}_2-\rvec{q}} 
		| \Psi^A_0 >
		\right|^2
	\nonumber \\
	&=&
		\left( G^p_E(q_\mu^2) \right)^2
	\nonumber \\
	&\times&
	\big[
		S( \rvec{p}'_1-\rvec{q}, \rvec{p}'_2, 
		   \rvec{p}'_1-\rvec{q}, \rvec{p}'_2 )
		+
		S( \rvec{p}'_1-\rvec{q}, \rvec{p}'_2, 
                   \rvec{p}'_1, \rvec{p}'_2-\rvec{q} )
	\nonumber \\
	&+&
		S( \rvec{p}'_1, \rvec{p}'_2-\rvec{q}, 
                   \rvec{p}'_1-\rvec{q}, \rvec{p}'_2 )
		+
		S( \rvec{p}'_1, \rvec{p}'_2-\rvec{q}, 
                   \rvec{p}'_1, \rvec{p}'_2-\rvec{q} )
	\big]
	\label{eq:spect}
	\end{eqnarray}
%
where $S$ is the spectral function (\ref{eq:CRSF}) 
of chapter \ref{chap:ppknock}.
